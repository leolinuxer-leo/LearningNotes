\documentclass[12pt]{article}
%\usepackage[utf8]{inputenc}
%\documentclass[UTF8]{ctexart}
%\usepackage[UTF8, heading = false, scheme = plain]{ctex}
\usepackage{geometry}
%geometry{a4paper,scale=0.9}
\geometry{a4paper,left=1cm,right=1cm,top=1cm,bottom=2cm}
\usepackage{amsfonts}
\usepackage{color}
\usepackage{url}
%\usepackage{biblatex}
\usepackage{amsmath}
\usepackage{amssymb}
\usepackage{latexsym}
\usepackage{cite}
%\addbibresource{ref.bib}
%\bibliography{ref.bib}
\usepackage{caption}
\usepackage{graphicx, subfig}
\usepackage{float}
%\usepackage[fontset=ubuntu]{ctex}
%\usepackage{fontspec}
\usepackage{xeCJK}
%\usepackage[colorlinks,
%anchorcolor=black,
%citecolor=black]{hyperref}
%\setmainfont{SimSun}
\usepackage[section]{placeins}
\usepackage{enumitem}
\usepackage{framed}
\usepackage[framemethod=TikZ]{mdframed}
\usepackage{indentfirst}
\usepackage{setspace}%使用间距宏包
\linespread{1.5}
%\title{预备知识}
%\author{leolinuxer }
%\date{June 2020}

\title{变分法\cite{Talk_About_Variational_Principle}}
%\author{leolinuxer }
%\date{June 2020}

\begin{document}
\maketitle

\section{简单的背景介绍}
让一个物体从静止开始沿着一个光滑无摩擦的轨道下滑,如果要求下滑过程耗时最短,轨道应该是什么形状?这个问题被称作最速降曲线问题(the brachistochrone problem)。

\section{变分大法}
假设我们有两个定点 $(a,p)$ 和 $(b,q)$,连接这两点的任意曲线的方程$y=f(x)$ 都将满足如下的边界条件:
\begin{equation}
    y(a) = p, y(b) = q
\end{equation}

现在考虑如下形式的定积分:
\begin{equation}
    I = \int_a^b{f(y,y')dx}
\end{equation}

其中$f(y,y')$ 是关于$y(x)$和其一阶导数 $y'(x)$ 的函数,我们期望找到一个具体的 $y(x)$,使得$I$ 有极值(极大或极小)。

注意在一般的极值问题中,我们考察的是自变量$x$ 的变化:$x$ 取值多少时,函数会有极值。而现在这个新问题的不同之处,我们考察的是函数$y(x)$的变化$y(x)$ 是什么形式时,$I$会有极值(高大上叫法:$I$ 称作函数 $y(x)$ 的泛函)。\textbf{然而这两类问题依然有共通之处:当 $I$ 取极值时,对 $y(x)$ 作微小的变化,$I$ 在一级近似下应该保持不变。}

如果$y(x)$有微小改变$\delta y(x)$(高大上叫法:$\delta y(x)$ 称作函数$y(x)$ 的变分),那么 $f(y,y')$ 的变化为:
\begin{equation}
    \delta f = \frac{\partial f}{\partial y} \delta y + \frac{\partial f}{\partial y'} \delta y'
\end{equation}

$I$ 相应的变化为:
\begin{equation}
    \delta I = \int_a^b[\frac{\partial f}{\partial y} \delta y + \frac{\partial f}{\partial y'} \delta y']dx
\end{equation}

方括号里的第二项可以改写成 $\frac{\partial f}{\partial y'}\frac{d(\delta y)}{dx}$,然后我们可以进行分部积分:

\begin{align}
    \int_a^b{\frac{\partial f}{\partial y'}\delta y'dx} &= \int_a^b\frac{\partial f}{\partial y'}d(\delta y) \\
    &= \frac{\partial f}{\partial y'}\partial y \big|_a^b - \int_a^b{\delta y}\frac{d}{dx}(\frac{\partial f}{\partial y'})dx
\end{align}

由于$y(x)$的边界条件固定,$\delta y(a) = \delta y(b) = 0$,所以分部积分出来的第一项为零,仅第二项有贡献。代回(4)式中,稍作化简可以得到
\begin{equation}
    \delta I = \int_a^b[\frac{\partial f}{\partial y} - \frac{d}{dx} (\frac{\partial f}{\partial y'})]\delta y(x)dx
\end{equation}

如果$I$ 有极值,对任意满足边界条件的 $\delta y(x)$都必须有$\delta I = 0$,这就要求:
\begin{equation}
   \frac{\partial f}{\partial y} - \frac{d}{dx} (\frac{\partial f}{\partial y'}) = 0
\end{equation}

这便是传说中的 Euler-Lagrange 方程,它是变分法的核心定理。有了此等大杀器,原则上就可以找出所寻求的极值函数$y(x)$。

通常来讲 Euler-Lagrange 方程会是一个二阶的微分方程, $y(x)$ 的通解中含有的两个待定常数刚好可以通过两个边界条件确定。我们下面来举几个例子操练操练。

\subsection{例:两点间的最短路经}
先来一个简单的例子小试牛刀。
给定平面上两点$(a,p)$和 $(b,q)$,连接它们的长度最短的曲线是什么?

曲线$y(x)$上相近的两点 $(x,y)$ 和$(x+dx,y+dy)$ 之间的曲线元长度为:
\begin{equation}
   ds = \sqrt{dx^2+dy^2} = \sqrt{1+y'^2}dx
\end{equation}
曲线的总长度为:
\begin{equation}
   S = \int_a^b{\sqrt{1+y'^2}}dx
\end{equation}

现在希望 $S$ 有最小值,我们可以取 $f(y,y') = \sqrt{1+y'^2}$,运用 Euler-Lagrange 方程来寻找可以让 $S$ 有极小的函数$y(x)$。注意到:
\begin{equation}
   \frac{\partial f}{\partial y} = 0, \quad \frac{\partial f}{\partial y'} = \frac{y'}{\sqrt{1+y'^2}}
\end{equation}

代回(6)式中,容易得到
\begin{equation}
    \frac{d}{dx}\frac{\partial f}{\partial y'} = \frac{y'}{\sqrt{1+y'^2}} = 0
\end{equation}

括号里这一大坨的导数为零,那么括号里这一大坨必然是一个常数,我们马上可以推出$y'(x)$ 也必然是一个常数。因此我们需要寻找的$y(x)$ 满足直线方程:
\begin{equation}
    y = kx + c
\end{equation}
斜率$k$ 和截距 $c$ 很容易通过边界点的坐标算出。由此我们证明了大家非常熟悉的结论:两点之间直线段的距离最短。

\subsection{例:最速降曲线}
问题在开篇的历史故事介绍中已经有提到,我们这里直接进入解答环节。

为方便起见,我们将坐标系的 $y$-轴搞成朝下的方向,斜向下的轨道可以由函数$y(x)$ 给出,其中轨道的起点和终点分别设为$(0,0)$ 和 $(a,b)$,我们来试求最速降曲线的函数式。

当物理下滑到 $(x,y)$ 位置时,它的速度大小可以根据能量守恒关系解出
$$
\frac{1}{2}mv^2 = mgy
$$
\begin{equation}
    v = \sqrt{2gy}
\end{equation}

而根据定义,速度大小等于单位时间内走过的轨道长度
\begin{equation}
    v = \frac{ds}{dt} = \sqrt{1+y'^2}\frac{dx}{dt}
\end{equation}

联立后有:
\begin{equation}
    dt = \sqrt{\frac{1+y'^2}{2gy}}dx
\end{equation}

积分后就可以得到总时间的表达式:
\begin{equation}
    T = \frac{1}{\sqrt{2g}}\int_0^a{\frac{1+y'^2}{y}dx}
\end{equation}

令 $f(y,y') = \frac{1+y'^2}{y}$,有:
$$
    \frac{\partial f}{\partial y} =  -\frac{1}{2}\sqrt{\frac{1+y'^2}{y^3}}  \quad \frac{\partial f}{\partial y'} = \frac{y'}{\sqrt{y(1+y'^2)}}
$$

丢回Euler-Lagrange式里面,我们可以得到这么一个初步的方程:

\begin{equation}
   \frac{1}{2}\sqrt{\frac{1+y'^2}{y^3}} + \frac{d}{dx}(\frac{y'}{\sqrt{y(1+y'^2)}}) = 0
\end{equation}

看到这种东西,要保持平静,铁了头往下算,要相信好多恶心的东西会神奇地同归于尽。

\begin{equation}
   \frac{1}{2}\sqrt{\frac{1+y'^2}{y^3}} + \frac{y''}{\sqrt{y(1+y'^2)}} - \frac{1}{2}\frac{y'^2}{\sqrt{y^3(1+y'^2)}} - \frac{y'^2y''}{\sqrt{y(1+y'^2)^3}}
\end{equation}

\begin{equation}
   \frac{1}{2}(1+y'^2)^2 + y(1+y'^2)y'' - \frac{1}{2}y'^2(1+y'^2) - yy'^2y'' = 0
\end{equation}

\begin{equation}
   \frac{1}{2}(1+y'^2) + yy'' = 0
\end{equation}

瞧,柳暗花明又一村。不过这还远没完,解这个二阶微分方程还需要一个骚操作。我们对上式乘上一个$2y'$:

\begin{equation}
   y'(1+y'^2) + 2yy'y'' = 0
\end{equation}
\begin{equation}
   \frac{d}{dx}[y(1+y'^2)] = 0
\end{equation}

它要等于零,方括号里那一坨等于常数就完儿事了。且让我们将这个常数写作 $k$:
$$
y(1+y'^2) = k
$$
$$
y'^2 = \frac{k}{y} - 1 = \frac{k-y}{y}
$$
$$
y' = \frac{dy}{dx} = \sqrt{\frac{k-y}{y}}
$$

原来的二阶微分方程降次变成了一阶,我们终于可以愉快地分离变量两边积分了:
$$
x = \int{dx} = \int{\sqrt{\frac{y}{k-y}}dy}
$$

后略。

\section{分部积分}
$$
\int{u(x)v'(x)}dx = u(x)v(x) - \int{u'(x)v(x)}dx
$$
也可以简写成:
$$
\int{udv} = uv - \int{vdu}
$$

%\printbibliography
\bibliography{../ref}
\bibliographystyle{IEEEtran}
\end{document}
