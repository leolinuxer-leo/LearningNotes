\documentclass[12pt]{article}
%\usepackage[utf8]{inputenc}
%\documentclass[UTF8]{ctexart}
%\usepackage[UTF8, heading = false, scheme = plain]{ctex}
\usepackage{geometry}
%geometry{a4paper,scale=0.9}
\geometry{a4paper,left=1cm,right=1cm,top=1cm,bottom=2cm}
\usepackage{amsfonts}
\usepackage{color}
\usepackage{url}
%\usepackage{biblatex}
\usepackage{amsmath}
\usepackage{amssymb}
\usepackage{latexsym}
\usepackage{cite}
%\addbibresource{ref.bib}
%\bibliography{ref.bib}
\usepackage{caption}
\usepackage{graphicx, subfig}
\usepackage{float}
%\usepackage[fontset=ubuntu]{ctex}
%\usepackage{fontspec}
\usepackage{xeCJK}
%\usepackage[colorlinks,
%anchorcolor=black,
%citecolor=black]{hyperref}
%\setmainfont{SimSun}
\usepackage[section]{placeins}
\usepackage{enumitem}
\usepackage{framed}
\usepackage[framemethod=TikZ]{mdframed}
\usepackage{indentfirst}
\usepackage{setspace}%使用间距宏包
\linespread{1.5}
%\title{预备知识}
%\author{leolinuxer }
%\date{June 2020}


\title{求解五次以下方程的方法\cite{From_Linear_Equation_To_Galois_Theory}}
\author{leolinuxer}
%\date{June 2020}

\begin{document}
\maketitle
\tableofcontents

\section{一次、二次、三次、四次方程的求解}
\subsection{一次方程}
一次方程:
$$
px + q = 0, p, q \in Z, p \ne 0
$$
的解为:
$$
x = -\frac{q}{p}
$$

\subsection{二次方程}
一般的二次方程形式为:
$$
ax^2 + bx + c = 0, a \ne 0
$$

方程的解为:
$$
x_{1,2} = \frac{-b \pm \sqrt{b^2 - 4ac}}{2a}
$$

\subsection{三次方程}
一般的三次方程形式为:
$$
az^3 + bz^2 + cz + d = 0, a \ne 0
$$

可以转化为:
$$
y^3 + ay^2 + by + c = 0
$$

这种最高次项的系数为1的方程,称为\textbf{首1(多项式)方程}。

对方程进行变量替换:$y = x - \frac{a}{3}$,有:
\begin{align*}
(x-\frac{a}{3})^3 + a(x-\frac{a}{3})^2 + b(x - \frac{a}{3}) + c &= 0 \\
x^3 - ax^2 + \frac{a^2}{3}x - \frac{a^3}{27} + ax^2 - \frac{2a^2}{3}x + \frac{a^3}{9} + bx - \frac{ab}{3} + c &= 0 \\
\end{align*}

也就是说,转换成了:
$$
x^3 + px + q = 0
$$
类型的方程。这是简化后的三次方程。

令 $x = u + v$,代入后化简可得:
$$
(u^3 + v^3) + 3uv(u+v) + p(u+v) + q = 0
$$

$x$是一个未知数,而 $u,v$ 是2个未知数,为此我们对 $u,v$ 再加一个约束条件:$3uv = -p$,于是上式可化简为:
$$
u^3 + v^3 + q = 0
$$

以$v = \frac{-p}{3u}$ 代入,可得:
$$
u^6 + qu^3 - \frac{p^3}{27} = 0
$$
$$
v^6 + qv^3 - \frac{p^3}{27} = 0
$$

因此$u,v$都是同一个六次方程的根。该六次方程的根很容易看出来是:
$$
u^3 = \frac{-q \pm \sqrt{q^2 + 4p^3/27}}{2} = -\frac{q}{2} \pm \sqrt{\frac{q^2}{4} + \frac{p^3}{27}}
$$

即原问题可求解,可得到卡丹公式如下:
$$
x = u+v = \sqrt[3]{\frac{-q}{2} + \sqrt{\frac{q^2}{4} + \frac{p^3}{27}}} + \sqrt[3]{\frac{-q}{2} - \sqrt{\frac{q^2}{4} + \frac{p^3}{27}}} 
$$

\begin{framed}  
%\verb|\documentstyle[ifthen,12pt,titlepage]{article}|
\small{
解方程:$x^3 - 15x - 126 = 0$

此时$p = -15, q = -126$,有$u^3 = 125$,\textbf{$u$有三个解:$u_1 = 5, u_2 = 5(\cos(120^\circ) + i\sin(120^\circ)), u_3 = 5(\cos(240^\circ) + i\sin(240^\circ))$},令$\cos(120^\circ) + i\sin(120^\circ) = \omega$,即$\omega = -\frac{1}{2} + \frac{\sqrt{3}}{2}i$,有$u_2 = 5\omega, u_3 = 5\omega^2$,于是从:$v = \frac{-p}{3u}$,有 $v_1 = 1, v_2 = \omega^2, v3 = \omega$,最后有:$x_{1,2,3} = 6, 5\omega+\omega^2, 5\omega^2 + \omega$。
}
\end{framed}

\subsection{四次方程}
对于一般首1的四次方程:
$$
y^4 + ay^3 + by^2 + cy + d = 0
$$

先进行变量替换 $y = x - \frac{a}{4}$,化为如下一般首1的简化四次方程:
$$
x^4 + px^2 + qx + r = 0
$$

令:
$$
x^4 + px^2 + qx + r = (x^2 + kx + l)(x^2 + nx + m)
$$

比较两边的各项系数,可以得到:
$$
n = -k, \quad l + m - k^2 = p, \quad k(m-l) = q, \quad lm = r
$$

如果能解出$k,l,m,n$,就可以通过求解两个二次方程来得到原四次方程的解。

为此,可以得到:
$$
2m = k^2 + p + q/k, \quad 2l = k^2 + p - q/k
$$

于是,有:
$$
k^6 + 2pk^4 + (p^2 - 4r)k^2 - q^2 = 0
$$

这是关于 $k^2$ 的三次方程,因此 $k$ 是可解的,于是 $m, l, n$ 也可解,于是得到 $x$ 的解。

\section{有关方程的一些理论}
\subsection{韦达与根和系数的关系}
\textbf{韦达定理}:若$a_1, a_2$是方程$x^2 + px + q=0$的两个根,则:$x^2 + px + q = (x - a_1)(x - a_2)$,从而有:
$$
a_1 + a_2 = -p, \quad a_1\cdot a_2 = q
$$

同理,对于三次方程:$x^3 + rx^2 + px + q = 0$,有:
$$
a_1 + a_2 + a_3 = -r, \quad a_1\cdot a_2 + a_2\cdot a_3 + a_3 \cdot a_1 = p, \quad a_1\cdot a_2 \cdot a_3 = -q
$$

更进一步,可以证明,如果$a_1, a_2, \cdots, a_n$ 是一般首1的$n$次方程 $x^n + b_1x^{n-1} + \cdots + b_n = 0$的 $n$ 个根,则有:
\begin{align*}
\sigma_1 &= a_1 + a_2 + \cdots + a_n = -b_1  \\
\sigma_2 &= a_1a_2 + a_1a_3 + \cdots + a_1a_n + a_2a_3 + \cdots + a_{n-1}a_n = b_2 \\
& \cdots \\
\sigma_n &= a_1a_2\cdots a_n = (-1)^nb_n
\end{align*}

这里的$\sigma_k, k = 1, 2, \cdots, n$ 表示所有可能的$k$个$a_i$ 的乘积之和。

\subsection{牛顿与牛顿定理}
对于两个变量$a_1,a_2$而言,表达式$\sigma_1 = a_1 + a_2, \sigma_2 = a_1 \cdot a_2$都是\textbf{对称多项式},因为它们在 $a_1$ 变为 $a_2$,$a_2$ 变为 $a_1$ 的同时置换下都保持不变。$a_1 \rightarrow a_2, a_2 \rightarrow a_1$ 可以形象地表示为:
$$
\begin{pmatrix}
a_1 & a_2 \\
\downarrow & \downarrow \\
a_2 & a_1
\end{pmatrix}
$$

或更简单的表示为:
$$
\begin{pmatrix}
a_1 & a_2 \\
a_2 & a_1
\end{pmatrix}
\text{或}
\begin{pmatrix}
1 & 2 \\
2 & 1
\end{pmatrix}
$$

当然它们在变换:
$$
\begin{pmatrix}
a_1 & a_2 \\
\downarrow & \downarrow \\
a_1 & a_2
\end{pmatrix}
\text{或}
\begin{pmatrix}
a_1 & a_2 \\
a_1 & a_2
\end{pmatrix}
\text{或}
\begin{pmatrix}
1 & 2 \\
1 & 2
\end{pmatrix}
$$

下也不变,容易看出$a_1^2 + a_2^2$ 在 
$$
S_2 = \begin{Bmatrix}
\begin{pmatrix}
1 & 2 \\
1 & 2 \\
\end{pmatrix}, & \begin{pmatrix}
1 & 2 \\
2 & 1 \\
\end{pmatrix} 
\end{Bmatrix}
$$
下也是不变的,因此$a_1^2 + a_2^2$ 也是对称多项式,进而从:
$$
a_1^2 + a_2^2 = (a_1 + a_2)^2 - 2a_1a_2 =   \sigma_1^2 - 2\sigma_2    
$$
可知,对称多项式$a_1^2 + a_2^2$可以用对称多项式$\sigma_1, \sigma_2$ 的多项式来表示,这似乎表明$\sigma_1, \sigma_2$更基本一些,为此,我们把$\sigma_1, \sigma_2$称为\textbf{基本对称多项式或初等对称多项式}。

对于3个变量$a_1, a_2, a_3$而言,表达式$\sigma_1 = a_1 + a_2 + a_3, \sigma_2 = a_1 \cdot a_2 + a_2 \cdot a_3 + a_3 \cdot a_1, \sigma_3 = a_1 \cdot a_2 \cdot a_3$  就是初等对称多项式,而在多项式$5a_1^3 + 5a_2^3 + 5a_3^3 - 15a_1a_2a_3$ 中,$a_1, a_2, a_3$的“地位”是完全一样的,因此它就是 $a_1, a_2, a_3$ 的对称多项式。用严格的数学语言来说,这指的是它在下面6个\textbf{变量}或\textbf{置换}下是\textbf{不变的}:
\begin{align*}
S_3 &= \begin{Bmatrix}
\begin{pmatrix}
1 & 2 & 3\\
1 & 2 & 3
\end{pmatrix}, & 
\begin{pmatrix}
1 & 2 & 3\\
1 & 3 & 2
\end{pmatrix}, &
\begin{pmatrix}
1 & 2 & 3\\
3 & 2 & 1
\end{pmatrix}, &
\begin{pmatrix}
1 & 2 & 3\\
2 & 1 & 3
\end{pmatrix}, &
\begin{pmatrix}
1 & 2 & 3\\
2 & 3 & 1
\end{pmatrix}, &
\begin{pmatrix}
1 & 2 & 3\\
3 & 1 & 2
\end{pmatrix}
\end{Bmatrix}  \\
&= \{g_1, g_2, g_3, g_4, g_5, g_6\}
\end{align*}

此外,从
$$5a_1^3 + 5a_2^3 + 5a_3^3 - 15a_1a_2a_3  = 5(a_1 + a_2 + a_3)^3 - 15(a_1+a_2+a_3)(a_1a_2 + a_2a_3 + a_3a_1) = 5\sigma_1 - 15\sigma_1\sigma_2$$  
可知,$5a_1^3 + 5a_2^3 + 5a_3^3 - 15a_1a_2a_3$可用初等多项式$\sigma_1, \sigma_2$的多项式表出。

\textbf{牛顿定理:任何一个关于变量$a_1, a_2, \cdots, a_n$的对称多项式,都可以唯一的表示为初等多项式$\sigma_1, \sigma_2, \cdots, \sigma_n$的一个多项式}。

\begin{framed}  
%\verb|\documentstyle[ifthen,12pt,titlepage]{article}|
\small{
一类常见的问题:不解方程 $x^2 + bx + c = 0$,求 $x_1^2 + x_2^2$ 的值

该问题的解题过程就是在验证牛顿定理;把 $x_1^2 + x_2^2$ 用初等对称多项式$\sigma_1 = x_1 + x_2 = -b, \sigma_2 = x_1 \cdot x_2 = c$ 表示出来
}
\end{framed}

\subsection{欧拉与复数}
复数的\textbf{代数表示式}:
$$
z = a + bi, \quad a,b \in R
$$

复数的\textbf{三角表示式}:
$$
z = r(\cos\theta + i\sin\theta)
$$

同时,有:
$$
e^{i\theta} = \cos\theta + i\sin\theta
$$

所以有复数的\textbf{指数表示式}:
$$
z = e^{i\theta}
$$

令$\theta=\pi$,可以得到\textbf{欧拉魔幻等式}:
$$
e^{i\pi} + 1 = 0
$$

利用指数表示式,有:
$$
e^{i\theta_1} \cdot e^{i\theta_2} = (\cos\theta_1 + i\sin\theta_1)\cdot(\cos\theta_2 + i\sin\theta_2) = e^{i(\theta_1+\theta_2)}
$$

同时,有:
$$
(e^{i\theta})^n = e^{in\theta}
$$

\subsection{1的根}
方程:
$$
x^n - 1 = 0
$$
的根,可以利用公式$(e^{i\theta})^n = e^{in\theta}$求得。设$x = r(\cos\theta + i\sin\theta)$,则从 $x^n = 1$ 可得,$r = 1$,且$n\theta = 2k\pi, k = 0, 1, \cdots, n-1$,所以方程的解为:
$$
1, \zeta = e^{i2\pi/n},  \zeta^2 = e^{i4\pi/n}, \cdots,  \zeta^{n-1} = e^{i2\pi(n-1)/n}
$$

一般来说,上述解是用指数式或三角式表示的,还不是根式解,不过,对于$n=1,2,3,4$,不难求得下列各解:
$$
1; 1, -1; 1, \omega, \omega^2; 1, i, -1, -i
$$
其中,$\omega = -\frac{1}{2} + \frac{\sqrt{3}}{2}i$,且满足:
$$
1 + \omega + \omega^2 = 0, \omega \cdot \omega^2 = \omega^3 = 1
$$

同时,因为 $x^n - 1= (x-1)(x^{n-1} + x^{n-2} + \cdots + x + 1)$,$\zeta$是$x^{n-1} + x^{n-2} + \cdots + x + 1 = 0$ 的根,也是 $x^n-1=0$的根,所以对于 $\zeta$,有:
$$
1 + \zeta + \zeta^2 + \cdots + \zeta^{n-1} = 0, \quad \zeta^i\cdot\zeta^{n-i} = 1, i = 1, 2, \cdots, n
$$

\section{范德蒙与“根的对称式表达”方法}
\subsection{范德蒙方法}
以方程 $x^2 + bx + c = 0$ 为例,设方程的根为:$a_1, a_2$,由$x^2=1$有解$\pm 1$,且$(+1) + (-1) = 0$,有:
$$
a_1 = \frac{1}{2}[(a_1 + a_2) + (a_1 - a_2)]
$$
$$
a_2 = \frac{1}{2}[(a_1 + a_2) - (a_1 - a_2)]
$$

其中,$a_1 + a_2 = \sigma = -b$ 是根的初等对称多项式,而$a_1 - a_2$却不是根的对称多项式,不过注意到
$$
[\pm(a_1 - a_2)]^2 = (a_1 + a_2)^2 - 4a_1a_2 = \sigma_1^2 - 4\sigma_2 = b^2 - 4c
$$
因此,$(b^2-4c)^{\frac{1}{2}} = \pm(a_1-a_2)$,如果 $b^2-4c \ge 0$,且符号$\sqrt{b^2-4c}$表示算术根的话,就可以得到:
$$
a_{1,2} = \frac{-b \pm \sqrt{b^2-4c}}{2a}
$$

\subsection{用范德蒙方法解三次方程}
设$x^3 + px + q = 0$的根为$a_1, a_2, a_3$,注意到$x^3=1$有三个根 $1, \omega, \omega^2$满足 $1+\omega+\omega^2=0$,$w^3 = 1$,则:
\begin{align*}
a_1 &= \frac{1}{3}[(a_1 + a_2 + a_3) + (a_1 + \omega a_2 + \omega^2a_3) + (a_1 + \omega^2 a_2 + \omega a_3)],\\
a_2 &= \frac{1}{3}[(a_1 + a_2 + a_3) + \omega^2(a_1 + \omega a_2 + \omega^2a_3) + \omega(a_1 + \omega^2 a_2 + \omega a_3)],\\
a_2 &= \frac{1}{3}[(a_1 + a_2 + a_3) + \omega(a_1 + \omega a_2 + \omega^2a_3) + \omega^2(a_1 + \omega^2 a_2 + \omega a_3)],\\
\end{align*}

\begin{framed}  
%\verb|\documentstyle[ifthen,12pt,titlepage]{article}|
\small{
推导思路:
\begin{align*}
a_i &= \frac{1}{3}[(a_i + a_i + a_i) + (1+\omega+\omega^2)a_j + (1+\omega+\omega^2)a_k]\\
    &= \frac{1}{3}[(a_i + a_j + a_k) + (a_i + \omega a_j + \omega^2 a_k) + (a_i + \omega^2a_j + \omega a_k)]
\end{align*}
}
\end{framed}

记
$$
U = (a_1 + \omega a_2 + \omega^2 a_3)^3, V = (a_1 + \omega^2 a_2 + \omega a_3)^3
$$

因此,这三个根可以统一写成:
$$
x = \frac{1}{3}(a_1+a_2+a_3) + \sqrt[3]{\frac{U}{27}} + \sqrt[3]{\frac{V}{27}}
$$

上一节中,我们想办法将 $a_1-a_2$ 联系了起来,我们接下来需要想办法将$U,V$和根$a_1,a_2,a_3$的初等对称多项式$\sigma_1, \sigma_2, \sigma_3$联系起来。

对 $U,V$施以$S_3$ 的各置换,有如下结果:
\begin{table}[H]
	\centering  % 显示位置为中间
	\begin{tabular}{|c|c|c|c|}  
	    \hline
	    置换 & 作用对象 & U & V \\
	    \hline
	    $g_1 = \begin{pmatrix}
	    1 & 2 & 3\\1 & 2 & 3
	    \end{pmatrix}$ & 得出结果 & $U$ & $V$ \\
		\hline
		$g_2 = \begin{pmatrix}
	    1 & 2 & 3\\1 & 3 & 2
	    \end{pmatrix}$ & 得出结果 & $V$ & $U$ \\
	    \hline
	    $g_3 = \begin{pmatrix}
	    1 & 2 & 3\\3 & 2 & 1
	    \end{pmatrix}$ & 得出结果 & $V$ & $U$ \\
	    \hline
	    $g_4 = \begin{pmatrix}
	    1 & 2 & 3\\2 & 1 & 3
	    \end{pmatrix}$ & 得出结果 & $V$ & $U$ \\
	    \hline
	    $g_5 = \begin{pmatrix}
	    1 & 2 & 3\\2 & 3 & 1
	    \end{pmatrix}$ & 得出结果 & $U$ & $V$ \\
	    \hline
	    $g_6 = \begin{pmatrix}
	    1 & 2 & 3\\3 & 1 & 2
	    \end{pmatrix}$ & 得出结果 & $U$ & $V$ \\
	    \hline
	\end{tabular}
\end{table}

\begin{framed}  
%\verb|\documentstyle[ifthen,12pt,titlepage]{article}|
\small{
推导思路:
\begin{align*}
    & g_1: \text{略} \\
    & g_2: \text{略} \\
    & g_3: \quad U' = (a_3+\omega a_2 + \omega^2 a_1)^3 = \omega^3 (\omega^2 a_1 + \omega a_2 + a_3)^3 = 
    (\omega^3 a_1 + \omega^2 a_2 + \omega a_3)^3 =
     (a_1 + \omega^2 a_2 + \omega a_3)^3 = V\\
     & g_4: \quad U' = (a_2+\omega a_1 + \omega^2 a_3)^3 = \omega^6 (\omega a_1 + a_2 + \omega^2a_3)^3 = 
    (\omega^3 a_1 + \omega^2 a_2 + \omega^4 a_3)^3 = (a_1 + \omega^2 a_2 + \omega a_3)^3 = V\\
    & g_5: \text{略} \\
    & g_6: \text{略} \\
\end{align*}
}
\end{framed}

由此可见,$U,V$在$S_3$ 变换下,都不是不变的,而 $U+V, U\cdot V$却是$S_3$下不变的。即它们是$a_1,a_2,a_3$的对称多项式,因此根据牛顿定律,可以用:
$$
\sigma_1 = a_1 + a_2 + a_3 = 0, \sigma_2 = a_1a_2 + a_2a_3 + a_3a_1 = p, \sigma_3 = a_1a_2a_3 = -q
$$
表出,经过一些代数运算后,有:
$$
U + V = -27q, U \cdot V = -27p^3
$$

因此,$U,V$是如下方程的根:
$$
t^2 + 27qt - 27p^3 = 0
$$

所以,可以解出来$U,V$,根据$x = \frac{1}{3}(a_1+a_2+a_3) + \sqrt[3]{\frac{U}{27}} + \sqrt[3]{\frac{V}{27}}$ 便能得到卡丹公式:
$$
x = \sqrt[3]{\frac{-q}{2} + \sqrt{\frac{q^2}{4}+ \frac{p^3}{27}}} + \sqrt[3]{\frac{-q}{2} + \sqrt{\frac{q^2}{4} - \frac{p^3}{27}}}
$$

\begin{framed}  
%\verb|\documentstyle[ifthen,12pt,titlepage]{article}|
\small{
$$
U = (a_1 + \omega a_2 + \omega^2 a_3)^3
$$
$$
V = (a_1 + \omega^2 a_2 + \omega a_3)^3
$$

因为$a_1, a_2, a_3$ 是方程$x^3+px+q=0$的解,所以$
a_1^3 + a_2^3+a_3^3 = -p(a_1+a_2+a_3)-3q = -3q$

因为$1 + \omega + \omega^2 = 0$,所以$\omega + \omega^2 = -1$

\begin{align*}
U + V &= 2(a_1^3 + a_2^3+a_3^3) + 12a_1a_2a_3\omega^3 \\
&+ 3(\omega a_1^2a_2
+ \omega^2a_1^2a_3 + \omega^4 a_2^2a_3 + \omega^2a_1a_2^2 + \omega^4a_1a_3^2 + \omega^5a_2a_3^2) \\
&+ 3(\omega^2 a_1^2a_2
+ \omega a_1^2a_3 + \omega^5 a_2^2a_3 + \omega^4a_1a_2^2 + \omega^2a_1a_3^2 + \omega^4a_2a_3^2) \\
&= -6q - 12q - 3(a_1^2a_2 + a_1^2a_3 + a_2^2a_3 + a_1a_2^2 + a_1a_3^2 + a_2a_3^2) \\
&= -18q - [(a_1+a_2+a_3)^3-(a_1^3+a_2^3+a_3^3) - 6abc] \\
&= -18q - 3q - 6q = -27q
\end{align*}

\begin{align*}
U \cdot V &= [(a_1+\omega a_2+\omega^2a_3)(a_1 + \omega^2a_2 + \omega a_3)]^3 \\
&= \cdots \\
&= [(a_1^2+a_2^2+a_3^2) - p]^3 \\
&= [(a_1+a_2+a_3)^2 - 2(a_1a_2 + a_1a_3 + a_2a_3) - p]^3 \\
&= [-2p - p]^3 \\
&= -27p^3
\end{align*}
}
\end{framed}

\section{拉格朗日和他的预解式方法}
略

\section{高斯与代数基本定理}
\subsection{代数基本定理}
$n(>0)$次多项式方程$x^n + a_1x^{n-1} + \cdots + a_{n-1}x + a_n = 0, a_i \in C, i = 1, 2, \cdots, n$ 有$n$个复数根。

如果我们一开始就在复数集合中求解方程,由于复系数方程的根仍还是复数,因此就不必再将复数集合扩张了。因此,$C$叫做\textbf{代数闭域}。

\subsection{分圆方程与它的根式求解}
略

\subsection{开方运算的多值性与卡丹公式}
略


%\printbibliography
\bibliography{../../ref}
\bibliographystyle{IEEEtran}
\end{document}