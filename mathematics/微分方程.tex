\documentclass[12pt]{article}
%\usepackage[utf8]{inputenc}
%\documentclass[UTF8]{ctexart}
%\usepackage[UTF8, heading = false, scheme = plain]{ctex}
\usepackage{geometry}
%geometry{a4paper,scale=0.9}
\geometry{a4paper,left=1cm,right=1cm,top=1cm,bottom=2cm}
\usepackage{amsfonts}
\usepackage{color}
\usepackage{url}
%\usepackage{biblatex}
\usepackage{amsmath}
\usepackage{amssymb}
\usepackage{latexsym}
\usepackage{cite}
%\addbibresource{ref.bib}
%\bibliography{ref.bib}
\usepackage{caption}
\usepackage{graphicx, subfig}
\usepackage{float}
%\usepackage[fontset=ubuntu]{ctex}
%\usepackage{fontspec}
\usepackage{xeCJK}
%\usepackage[colorlinks,
%anchorcolor=black,
%citecolor=black]{hyperref}
%\setmainfont{SimSun}
\usepackage[section]{placeins}
\usepackage{enumitem}
\usepackage{framed}
\usepackage[framemethod=TikZ]{mdframed}
\usepackage{indentfirst}
\usepackage{setspace}%使用间距宏包
\linespread{1.5}
%\title{预备知识}
%\author{leolinuxer }
%\date{June 2020}

\title{微分方程}
\author{leolinuxer}
%\date{June 2020}

\begin{document}
\maketitle
\section{基本概念及分类\cite{Differential_Equation_1_Basic}}
\subsection{基本概念}
微分方程指的是:含有未知函数及其导数的方程。

常微分方程(Ordinary Differential Equation,ODE)指的是:仅含有一个独立变量的微分方程。

偏微分方程指的是:微分方程中的未知函数包含两个或两个以上的独立变量。

微分方程的阶数取决于方程中出现的最高次导数阶数。

特解指的是满足微分方程的某一个解;通解指的是满足微分方程的一组解。一些微分方程有无穷多解,而有的微分方程无解,有的微分方程则仅有有限个解。

\subsection{初值问题和边界值问题}
在给微分方程添加附加条件时,如果附加条件中未知函数及其导数的独立变量取值相同,则为初值问题;如果附加条件中未知函数及其导数的独立变量取值不同,则为边界值问题。

一个初值问题或边界值问题的解 $y(x)$ 不仅要满足微分方程,还要满足所有附加条件。

\subsection{一阶微分方程的常见形式}
\subsubsection{标准及微分形式
}
一阶微分方程的标准形式是:
$$
y' = f(x,y)
$$

其中,微分部分仅出现在方程的左侧。大部分(不是所有)一阶微分方程都可以通过代数方法写为以上形式。

上式的右侧也可以写为两个函数$M(x,y)$和 $-N(x,y)$ 的商的形式。那么整体就可写为微分形式:
$$
\frac{dy}{dx} = \frac{M(x,y)}{-N(x,y)}
$$
$$
M(x,y)dx + N(x,y)dy = 0
$$

\subsubsection{线性方程}
对于一阶方程的标准形式,如果标准形式的右侧$f(x,y)$可以写为:
$$
f(x,y) = -p(x)y + q(x)
$$

即,一个关于 $x$ 的函数乘以 $y$  ,再加上一个关于  $x$  的函数。那么该微分方程即为线性微分方程,一阶线性微分方程可以写为以下形式:
$$
y' + p(x)y = q(x)
$$

\subsubsection{伯努利方程}
伯努利方程是具有以下形式的微分方程:
$$
y' + p(x)y = q(x)y^n
$$

其中, $n$ 为实数。当 $n=0$或 $n=1$ 时,伯努利方程将退化为线性方程形式。

\subsubsection{齐次方程}
对于一阶微分方程,如果对于任意实数 [公式] 满足:
$$
f(tx,ty) = f(x,y)
$$
则为齐次方程。注意:此处的“齐次”概念狭义上仅针对一阶微分方程成立,且与齐次线性微分方程中的“齐次”并非同一概念,注意区分。

\subsubsection{可分离变量方程}
对于前面提到的微分方程的微分形式:
$$
M(x,y)dx + N(x,y)dy = 0
$$

如果其满足 $M(X,y)=M(x)$(即为只与 [公式] 有关的函数)和 $N(X,y)=N(y)$,(即为只与 [公式] 有关的函数)。那么该微分方程即为可分离变量的微分方程:
$$
M(x)dx + N(y)dy = 0
$$

\subsubsection{恰当方程}
对于微分方程的微分形式:
$$
M(x,y)dx + N(x,y)dy = 0
$$

如果满足以下条件:
$$
\frac{\partial M(x,y)}{\partial y} = \frac{\partial N(x,y)}{\partial x}
$$
则称其为恰当方程。

\section{一阶常微分方程的解法\cite{Differential_Equation_2_Solution}}
\subsection{可分离变量方程 Separable Equations}
对于一阶可分离变量的微分方程:
$$
M(x)dx + N(y)dy = 0
$$

为求其解,只需两端积分:
$$
\int M(x)dx + \int N(y)dy = 0
$$
其中, $C$ 代表任意常数。

在实际情况中,上式的积分结果往往无法得到解析表达式。因此,通常采用数值方法得到近似解。就算上式的积分结果可以得到解析表达式,也可能得不到$y$ 关于变量 $x$ 的显式表达式,那么解将保留隐式表达式。

\begin{framed}  
%\verb|\documentstyle[ifthen,12pt,titlepage]{article}|
例子:求解
$$
\frac{dy}{dx} = \frac{x^2+2}{y}
$$

此方程可以写为以下形式:
$$
(x^2+2)dx - ydy = 0
$$
$$
\int(x^2+2)dx - \int ydy = C
$$
$$
\frac{1}{3}x^3 + 2x - \frac{1}{2}y^2 = C
$$

为了求解 $y$,可以先得到解的隐形表达式:
$$
y^2 = \frac{2}{3}x^3 + 4x + k \quad k = -2C
$$

所以,求解得到 $y$ 的显式表达式:
$$
y = \pm\sqrt{\frac{2}{3}x^3 + 4x + k}
$$
\end{framed}

\subsection{齐次方程 Homogeneous Equations}
\textbf{【注意】这里的“齐次”与一般的线性齐次微分方程中的“齐次”不是同一个概念,注意区分。}

对于齐次微分方程:
$$
\frac{dy}{dx} = f(x,y)
$$

其具有以下特性:
$$
f(tx,ty) = f(x,y)
$$

那么其可以通过代换令$y = xv$,其中,$v$ 也是关于 $x$的函数,使之变为可分离变量的微分方程:
$$
\frac{dy}{dx} = v + x\frac{dv}{dx}
$$

这样,可以用解可分离变量的方法得到变量$v$ 与变量 $x$ 之间的关系。再通过“逆代法”求解原方程。

或者,将原式写为:
$$
\frac{dx}{dy} = \frac{1}{f(x,y)}
$$

并令 $x = yu$ ,相应的微分为:
$$
\frac{dx}{dy} = u + y\frac{du}{dy}
$$
通过简化得到可分离变量形式。

\begin{framed}  
%\verb|\documentstyle[ifthen,12pt,titlepage]{article}|
例子:求解
$$
y' = \frac{y+x}{x}
$$

该微分方程不是可分离变量的。但它具有以下形式:
$$
y' = f(x,y)
$$
其中:
$$
f(x,y) = \frac{y+x}{x}
$$

而且满足:
$$
f(tx,ty) = f(x,y)
$$

因此,其为齐次微分方程。令 $y = xv$,得到:
$$
v + \frac{dv}{dx} = \frac{xv+x}{x}
$$
$$
x\frac{dv}{dx} = 1
$$
$$
\frac{1}{x}dx - dv = 0
$$

此时式子就是可分离变量的微分方程了。解为:
$$
\int\frac{1}{x}dx - \int vdv = C
$$

因此有:
$$
v = \ln|x| - C \rightarrow v = \ln|kx|
$$
其中,$C = -\ln|k|$

然后通过逆代法,$v = \frac{y}{x}$,则有:
$$
y = x\ln|kx|
$$
\end{framed}

\subsection{恰当方程 Exact Equations}
对于以下形式的微分方程:
$$
M(x,y)dx + N(x,y)dy = 0
$$

如果存在函数$g(x,y)$满足:
$$
dg(x,y) = M(x,y)dx + N(x,y)dy
$$
则称为“恰当方程”。

注意,可以验证:

如果$M(x,y)$和$N(x,y)$都是连续函数且在 $xy$ 平面上的具有一阶连续偏导。当且仅当下式成立时原方程为恰当方程:
$$
\frac{\partial M(x,y)}{\partial y} = \frac{\partial N(x,y)}{\partial x}
$$

假设原方程是恰当方程,即存在$g(x,y)$满足:
$$
\frac{\partial g(x,y)}{\partial x} = M(x,y)
$$
$$
\frac{\partial g(x,y)}{\partial y} = N(x,y)
$$

又由于,原方程是:
$$
M(x,y)dx + N(x,y)dy = 0
$$

即:
$$
dg(x,y(x)) = 0
$$
$$
\int dg(x,y(x)) = C
$$

\subsubsection{积分因子}
通常而言,原方程不是恰当方程。但是在某种特殊情形下却可以转化为满足恰当方程条件的微分方程。令积分因子 $I(x,y)$ ,如果能够使得下式成为恰当方程。则原方程的解可由下式得到:
$$
I(x,y)[M(x,y)dx + N(x,y)dy] = 0
$$

如果有:
$$
\frac{1}{N}(\frac{\partial M}{\partial y} - \frac{\partial N}{\partial x}) \equiv g(x)
$$

即结果仅是$x$的函数。那么:
$$
I(x,y) = e^{\int g(x)dx}
$$

如果有:
$$
\frac{1}{M}(\frac{\partial M}{\partial y} - \frac{\partial N}{\partial x}) \equiv h(x)
$$

即结果仅是$y$的函数。那么:
$$
I(x,y) = e^{-\int h(y)dy}
$$

如果有:
$$
M = yf(xy), N = xg(xy)
$$

那么:
$$
I(x,y) = \frac{1}{xM - yN}
$$

通常该积分因子很难找到,如果微分方程不满足上面的可能情况,则需要采用其他方法进行求解。

\begin{framed}  
%\verb|\documentstyle[ifthen,12pt,titlepage]{article}|
例子:求解
$$
2xydx + (1+x^2)dy = 0
$$

令 $M(x,y) = 2xy, N(x,y) = 1+x^2$,则有:
$$
\frac{\partial M}{\partial y} = \frac{\partial N}{\partial x} = 2x
$$

则该微分方程为恰当方程。因为该方程是恰当方程,因此,令函数$g(x,y)$满足上式。

对于$M(x,y) = 2xy$,由于 $\frac{\partial g(x,y)}{\partial x} = M(x,y)$

因此有:
$$
\int \frac{\partial g}{\partial x}dx = \int 2xydx
$$
$$
g(x,y) = x^2y+h(y)
$$

注意:这里仅针对$x$ 进行积分,所以常数部分可以与 $h$ 相关。

接着是要确定$h(y)$。

对于$N(x,y) = (1+x^2)$,由 $\frac{N(x,y)}{dy} = x^2 + h'(y)$ ,因此有:$h'(y) = 1$

所以有$h(y) = y + C_1$。

则有:
$$
g(x,y) = x^2y + y + C_1
$$

令 $g(x,y) = C$,得到微分方程的隐式解:
$$
x^2y + y = C_2 \quad (C_2 = C - C_1)
$$

显式解为:
$$
y = \frac{C_2}{x^2 + 1}
$$
\end{framed}

\begin{framed}  
%\verb|\documentstyle[ifthen,12pt,titlepage]{article}|
例子:确定下列微分方程是否为恰当方程:
$$
ydx - xdy = 0
$$

由微分形式微分方程可令:$M(x,y) = y$ 和 $N(x,y) = -x$,因此:
$$
\frac{\partial M}{\partial y} = 1 \neq \frac{\partial N}{\partial x} = -1
$$

因此,该微分方程不是恰当方程。
\end{framed}

\begin{framed}  
%\verb|\documentstyle[ifthen,12pt,titlepage]{article}|
例子:判断 $-\frac{1}{x^2}$ 是否为以下微分方程的积分因子:
$$
ydx - xdy = 0
$$

从上面的例子可知,原方程不是恰当方程,但是通过乘以 $-\frac{1}{x^2}$ ,可得:
$$
\frac{1}{x^2}(ydx - xdy) = 0
$$
$$
-\frac{y}{x^2}dx + \frac{1}{x}dy = 0
$$

此时有:$M(x,y) = -\frac{y}{x^2}$和$N(x,y) = \frac{1}{x}$
$$
\frac{\partial M}{\partial y} = -\frac{1}{x^2} = \frac{\partial N}{\partial x}
$$

即,转化后的方程为恰当方程,所以 $-\frac{1}{x^2}$是原微分方程的积分因子。
\end{framed}

\subsection{线性微分方程 Linear Equations}
一阶线性微分方程具有以下形式:
$$
y' + p(x)y = q(x)
$$

可以先将该标准形式转化为微分形式看看:
$$
[p(x)y - q(x)]dx + dy = 0
$$

然后,看看是否能转化为恰当方程。
$$
\frac{1}{N}(\frac{\partial M}{\partial y} - \frac{\partial N}{\partial x}) = p(x)
$$

发现结果仅仅与$x$ 有关,所以一阶线性微分方程都可以转化为恰当方程。

\section{伯努利方程 Bernoulli Equations}
伯努利方程具有以下形式:
$$
y' + p(x)y = q(x)y^n
$$
其中, $n$为实数。令:
$$
z = y^{1-n}
$$

这样就将原伯努利方程转化成了关于函数 $z(x)$的一阶线性常微分方程。就可用求解线性微分方程的方法来求解原方程。
%\printbibliography
\bibliography{../ref}
\bibliographystyle{IEEEtran}
\end{document}
