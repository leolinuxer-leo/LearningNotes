\documentclass[12pt]{article}
%\usepackage[utf8]{inputenc}
%\documentclass[UTF8]{ctexart}
%\usepackage[UTF8, heading = false, scheme = plain]{ctex}
\usepackage{geometry}
%geometry{a4paper,scale=0.9}
\geometry{a4paper,left=1cm,right=1cm,top=1cm,bottom=2cm}
\usepackage{amsfonts}
\usepackage{color}
\usepackage{url}
%\usepackage{biblatex}
\usepackage{amsmath}
\usepackage{amssymb}
\usepackage{latexsym}
\usepackage{cite}
%\addbibresource{ref.bib}
%\bibliography{ref.bib}
\usepackage{caption}
\usepackage{graphicx, subfig}
\usepackage{float}
%\usepackage[fontset=ubuntu]{ctex}
%\usepackage{fontspec}
\usepackage{xeCJK}
%\usepackage[colorlinks,
%anchorcolor=black,
%citecolor=black]{hyperref}
%\setmainfont{SimSun}
\usepackage[section]{placeins}
\usepackage{enumitem}
\usepackage{framed}
\usepackage[framemethod=TikZ]{mdframed}
\usepackage{indentfirst}
\usepackage{setspace}%使用间距宏包
\linespread{1.5}
%\title{预备知识}
%\author{leolinuxer }
%\date{June 2020}

\title{概率统计基础知识\cite{Common_Probability_Distribution}\cite{Inference_Of_Expectation_Variations}}
%\author{leolinuxer }
%\date{June 2020}

\begin{document}
\maketitle

\section{基础定义}
\subsection{统计相关的基础知识}
横断面研究(cross-sectional study):研究某个时间点下样本的情况

纵贯研究(longitudinal study):在一段时间内反复观察同一批样本\cite{Think_Stats}

\subsection{随机变量}
随机变量是可以随机地取不同值的变量。例如:抛掷一枚硬币,出现正面或者反面的结果;从一个班级中,随机选一个同学,他的身高值。

\textbf{随机变量可以是离散的或者连续的}。离散随机变量拥有有限(例如正面或者反面)或者可数无限多的状态。连续随机变量伴随着实数值(例如:身高)。

\subsection{概率质量函数和概率密度函数}
\textbf{离散型变量的概率分布可以用概率质量函数(Probability Mass Function,简称PMF)来描述}。我们通常用大写字母P来表示概率质量函数。

\textbf{当研究的对象是连续型随机变量时,我们用概率密度函数(Probability Density Function,简称PDF)来描述它的概率分布}。我们通常用小写字母p来描述概率密度函数。

\subsection{期望}
函数 $f(x)$ 关于某分布的期望(Expectation)或者期望值(Expected value)是指:当x由P产生,f作用于x时,$f(x)$的平均值。

对于离散型随机变量,期望值可以通过求和得到:
$$
E_{x \sim P[f(x)]} = \sum_xP(x)f(x)
$$

对于连续型随机变量,期望值可以通过求积分得到:
$$
E_{x \sim p[f(x)]} = \int p(x)f(x) dx
$$

\subsection{方差与标准差}
方差(Variance)衡量的是当我们对x依据它的概率分布进行采样时,随机变量x的函数值会呈现多大的差异。计算方差的函数如下:
$$
Var(f(x)) = E[(f(x)-E[f(x)])^2]
$$
方差的平方根称之为标准差(Standard Deviation)。有些资料也称标准差为均方差。

设有一个随机变量$X$, 其期望存在为$E(X)$,方差存在为$D(X)$,那么有结论:
$$
D(X) = E(X^2) - [E(X)]^2
$$

其中,$E(X^2)$ 是 $X^2$ 的期望。

例如,已知 $P(X=1)= 2/3, P(X=0) = 1/6, P(X=-1) = 1/6$,那么:

$$
E(X) = 1 * 2/3 + 0 * 1/6 +(-1) * 1/6 = 2/3 - 1/6 = 1/2
$$

$$
E(X^2) = 1^2 * 2/3 + 0^2 * 1/6 + (-1)^2 * 1/6 = 2/3 + 1/6 = 5/6.
$$

$$
D(X) = E(X^2) - [E(X)]^2 = 5/6 - (1/2)^2 = 7/12
$$


\subsubsection{一幅图理解期望和方差}
台湾大学李毅宏教授的这节课程: 《Where does the error come from?》 仅仅通过一幅图就非常好的解释了期望和方差的概念。

结果与期望的偏离称之为偏差(Bias)。通俗的讲,Bias描述了结果与中心的偏离程度。而方差(Variance)描述了结果互相之间的散列程度。

你可以对比下面这幅图的四种情况来加深理解:
\begin{figure}[ht]
  \centering
  \includegraphics[width=.8\textwidth]{fig/bias_variance.png} %1.png是图片文件的相对路径
  \caption{Bias 和方差的比较} %caption是图片的标题
  %\label{HingeLossExample} %此处的label相当于一个图片的专属标志,目的是方便上下文的引用
\end{figure}

从这幅图中我们可以看出:
\begin{itemize}
    \item 当Bias和Variance都比较小的时候,结果都比较紧密的集中在预期的值上。
    \item 当Bias较小而Variance较大时,意味着结果较靠近预期,但是比较散落。
    \item 当Bias较大而Variance较小时,意味着结果集中在一起,但是离预期值偏离较多。
    \item 当Bias和Variance都较大时,意味着结果既不靠近预期,也比较散落。
\end{itemize}

\subsubsection{凸函数和Jensen不等式\cite{Understanding_Jensen_Nnequality_And_Proof}}
\textbf{凸函数}

凸函数是一个定义在某个向量空间的凸子集 $C$(区间)上的实值函数 $f$,如果在其定义域 $C$ 上的任意两点 $x1,x2x $, $0 \le t \le 1 $ ,有
$$
tf(x_1) + (1-t)f(x_2) \ge f(tx_1 + (1-t)x_2)
$$

就是说凸函数任意两点的割线位于函数图形上方, 这也是\textbf{Jensen不等式的两点形式}。

\textbf{Jensen不等式}

若对于任意点集$\{xi\}$,若$\lambda_i \ge 0$ 且$\sum_i\lambda_i = 1$,则凸函数 $f(x)$ 满足:
$$
f(\sum_{i=1}^{M}\lambda_ix_i) \le \sum_{i=1}^{M}\lambda_if(x_i)
$$

\begin{framed}  
%\verb|\documentstyle[ifthen,12pt,titlepage]{article}|
\small{
使用数学归纳法证明如下:

当 $i=1$或 $i=2$时,根据凸函数的定义,显然成立;

假设当 $i=M$ 时不等式成立,现在证明当 $i=M+1$ 时不等式也成立:
\begin{align*}
f(\sum_{i=1}^{M+1}\lambda_ix_i) &= f(\lambda_{M+1}x_{M+1} + \sum_{i=1}^M\lambda_ix_i) \\
&= f(\lambda_{M+1}x_{M+1} + (1-\lambda_{M+1})\sum_{i=1}^M\eta_ix_i)
\end{align*}
其中,
$$
\eta_i = \frac{\lambda_i}{1 - \lambda_{M+1}}
$$

注意到 $\lambda_i$ 满足:
$$
\sum_{i=1}^{M+1}\lambda_i = 1
$$

所以:
$$
\sum_{i=1}^{M}\lambda_i = 1 - \lambda_{M+1}
$$

所以$\eta_i$ 满足:
$$
\sum_{i=1}^{M}\eta_i = \frac{\sum_{i=1}^{M}\lambda_i }{1 - \lambda_{M+1}} = 1
$$

所以:
$$
\sum_{i=1}^{M}f(\eta_ix_i) \le \sum_{i=1}^{M}\eta_if(x_i)
$$

所以命题得证:
$$
f(\sum_{i=1}^{M+1}\lambda_ix_i) \le \lambda_{M+1}f(x_{M+1}) + (1-\lambda-{M+1})\sum_{i=1}^{M}\eta_if(x_i) = \sum_{i=1}^{M+1}\lambda_if(x_i)
$$
}
\end{framed}

\subsubsection{Jensen不等式和期望}
\textcolor{red}{在概率论中,如果把$λ_i$ 看成取值为 $x_i$的离散变量 $x$ 的概率分布,那么 Jensen不等式就可以写成:}
$$
f(E[x]) \le E[f(x)]
$$

对于连续变量,Jensen不等式给出了积分的凸函数值和凸函数的积分值间的关系:
$$
f(\int{xp(x)dx}) \le \int{f(x)p(x)dx}
$$

\begin{framed}  
%\verb|\documentstyle[ifthen,12pt,titlepage]{article}|
\small{
一道关于概率论的不等式问题:已知 $X_1, X_2, X_3>0$是某个概率空间上的随机变量,证明:
$$
E[\frac{X_1}{X_2}]E[\frac{X_2}{X_3}]E[\frac{X_3}{X_1}] \ge 1
$$

证明:先看有两个变量的情况:

令$g(x) = \frac{1}{x}, \quad x > 0$,则 $g(x)$是右半空间上的凸函数。所以根据 Jensen 不等式,有:
$$
E[g(x)] \ge g(E[x]) = \frac{1}{E[x]}
$$

令 $x = \frac{X_1}{X_2}$,则有:
$$
E[g(\frac{X_1}{X_2})] \ge g(E[\frac{X_1}{X_2}]) = \frac{1}{E[\frac{X_1}{X_2}]}
$$
$$
E[\frac{X_1}{X_2}]E[g(\frac{X_1}{X_2})] \ge E[\frac{X_1}{X_2}]\frac{1}{E[\frac{X_1}{X_2}]} = 1
$$

更进一步,有(??如何证明??):
$$
E[\frac{X_1}{X_3}] = E[\frac{X_1}{X_2}\frac{X_2}{X_3}] \le E[\frac{X_1}{X_2}]E[\frac{X_2}{X_3}]
$$

那么对于三个元的情况类似得可以按以下方式处理,
$$
E[\frac{X_1}{X_2}]E[\frac{X_2}{X_3}]E[\frac{X_3}{X_1}] \ge E[\frac{X_1}{X_3}]E[\frac{X_3}{X_1}] = \ge 1
$$
}
\end{framed}

\subsubsection{期望和协方差的关系}
\begin{align*}
Cov(X,Y) &= E[X - E[X]]\cdot E[Y-E[Y]] \\
    &= E[XY] - E[XE[Y]+YE[X]] + E[E[X]E[Y]] \\
    &= E[XY] - 2E[X]E[Y] + E[X]E[Y] \\
    &= E[XY] - E[X]E[Y]
\end{align*}

\subsubsection{偏度(skewness)和皮尔逊中值偏度系数(Pearson's median skewness coefficient)\cite{Think_Stats}}
\textbf{偏度}是度量分布函数不对称程度的统计量。对于一个给定的序列 $x_i$,样本偏度的定义为:
$$
g_1 = m_3 / m_2^{\frac{3}{2}}
$$
$$
m_2 = \frac{1}{n}\sum_i(x_i - \mu)^2
$$
$$
m_3 = \frac{1}{n}\sum_i(x_i - \mu)^3
$$

负的偏度表示分布向左偏,此时分布函数的左边会比右边延伸的更长;正的偏度表示分布函数向右偏。

偏度受异常值的影响比较大。更好的比较方式是比较均值和中位数的大小。因为均值更容易受到极端值的影响,但是中位数不易受到影响。

\textbf{皮尔逊中值偏度系数}的定义如下:
$$
g_p = 3(\mu - \mu_{\frac{1}{2}})/\sigma
$$

其中,$\mu$ 为均值,$\mu_{\frac{1}{2}}$ 为中位数。

皮尔逊中值偏度系数是偏度的一个鲁邦估计,对异常值的影响不敏感。

\subsection{相关性}
\subsubsection{协方差(Covariance)\cite{Think_Stats}}
协方差用来衡量相关变量的变化趋势是否相同。假设我们有两列序列 $X$ 和 $Y$,他们与其均值的离差为:
$$
dx_i = x_i - \mu_x 
$$
$$
dy_i = y_i - \mu_y
$$

协方差就是这些乘积结果的平均值:
$$
Cov(X,Y) = \frac{1}{n}\sum{dx_idy_i}
$$

$n$ 表示序列的长度($X$和$Y$必须有相同的长度)。

\subsection{皮尔逊相关系数}
$$
p_i = \frac{(x-\mu_x)}{\sigma_x}\frac{(y_i-\mu_y)}{\sigma_y}
$$

皮尔逊相关系数定义为:
$$
\rho = \frac{1}{n}\sum{p_i}
$$

\textbf{相关系数$\rho$的取值为-1到1之间},简单的证明如下:
$$
\rho = \frac{1}{n}\sum{p_i} = \frac{Cov(X,Y)}{\sigma_X\sigma_Y}
$$

将离差项带入公式,有
$$
\rho = \frac{\sum dx_idy_i}{\sum{dx_i}\sum{dy_i}}
$$

利用著名的柯西-施瓦兹不等式(Cauchy-Schwarz innequality)即可证明 $\rho^2 <= 1$,即 $-1 <= \rho <= 1$

$\rho$描述了两个变量相关的程度,当 $\rho = 1$时,两个变量完全相关,即如果我们知道了其中一个变量的值,就可以精确预测另一个变量的值。$\rho = -1$时表示两个变量是完全负相关的。

皮尔逊相关系数受异常值的影响比较大。

\section{离散型分布}
\subsection{伯努利分布(零一分布)}
伯努利(Bernoulli)分布是最基本,也是我们最常见的分布。伯努利分布在生活中非常的常见。例如,抛掷一枚硬币的结果就是符合伯努利分布的:结果只会是正面或者反面。

伯努利分布亦称“零一分布”、“两点分布”:它的结果只会是两种可能性中的一种,且这两种结果互相对立,必居其一。因此,我们也常常称结果是成功的,或者失败的。

\textbf{好几种其他的分布都与伯努利分布有一定的关系。}

假设伯努利实验成功的概率是p,则伯努利分布的概率质量函数如下:
$$
P(X=x) = p^x (1 - p)^{(1-x)},\; X \in {0, 1}
$$

当然,考虑到X只有0和1两种可能,我们也可以直接写成:
$$
P(X=x) =
\begin{cases}
p\;(x=1) \\
1-p\;(x=0)
\end{cases}
$$

伯努利分布的期望值是 $p$,方差是 $p(1−p)$。

\subsection{二项分布}
我们可以很自然的可以将一次伯努利实验扩展到多次,此时其结果符合二项(Binomial)分布。

二项分布的概率质量函数如下:
$$
P(X=k) = \binom{n}{k}p^{k}(1-p)^{n-k}
$$
$$
\binom{n}{k}  = \frac{n!}{k!(n-k)!}
$$

通过这个函数,我们可以计算出在进行n次的伯努利实验中,有k次出现正面结果的概率。

很显然,当n为1时,这个函数和伯努利的概率质量函数是一样的。

二项分布的期望值是$np$,方差是$np(1-p)$。

期望的推导过程:
\begin{align}
    E(X) &= \sum_{k=0}^nP\{X=k\}k \\
    &= \sum_{k=1}^n \binom{n}{k}kp^{k}(1-p)^{n-k} \\
    &= \sum_{k=1}^nnp\binom{n-1}{k-1}p^{k-1}(1-p)^{n-k} \\
    &= np\sum_{k=1}^n\binom{n-1}{k-1}p^{k-1}(1-p)^{n-k} \\
    &= np
\end{align}
这里有两个变换:
$$
\binom{n}{k} = \frac{n!}{k!(n-k)!} = \frac{n}{k}\binom{n-1}{k-1}
$$
$$
\sum_{k=1}^n\binom{n-1}{k-1}p^{k-1}(1-p)^{n-k} = (p+1-p)^{(n-1)} = 1
$$
\begin{framed}  
%\verb|\documentstyle[ifthen,12pt,titlepage]{article}|
附:该式的推导,已知二次函数
$$
(a+b)^{n-1} = \sum_{i=0}^{n-1}\binom{n-1}{i}b^{n-1-i}a^i
$$

可以看出来,$i \in \{0, 1, \cdots, n-1\}$

令 $k=i+1$,所以 $k \in \{1, 2, \cdots, n\}$,且有:
$$
(a+b)^{n-1} = \sum_{k=1}^{n-1}\binom{n-1}{k-1}b^{n-k}a^{k-1}
$$

进一步可以推导出:
$$
(a+b)^{n-2} = \sum_{k=2}^{n-2} \binom{n-2}{k-2}b^{n-k}a^{k-2}
$$
\end{framed}  

方差的推导过程:
\begin{align}
    E(X^2) &= \sum_{k=0}^n k^2\binom{n}{k}p^k(1-p)^{n-k}  \quad \text{令 1-p = q}\\
    E(X^2) &= \sum_{k=1}^n k^2\binom{n}{k}p^kq^{n-k} \\
    &= \sum_{k=1}^n nk\binom{n-1}{k-1}p^kq^{n-k} \\
    &= \sum_{k=1}^n n(k-1+1)\binom{n-1}{k-1}p^kq^{n-k} \\
    &= \sum_{k=1}^n n(k-1)\binom{n-1}{k-1}p^kq^{n-k} + \sum_{k=1}^n np\binom{n-1}{k-1}p^{k-1}q^{n-k} \\
    &= \sum_{k=2}^n n(n-1)p^2\binom{n-2}{k-2}p^{k-2}q^{n-k} + np \\
    &= n(n-1)p^2 + np
\end{align}

\begin{align}
    D(X) &= E(X^2) - [E(X)]^2 \\
         &= n(n-1)p^2 + np - (np)^2 \\
         &= n^2p^2 - np^2 + np - n^2p^2\\
         &= np - np^2 \\
         &= np(1-p)
\end{align}


\subsection{几何分布}
几何(Geometric)分布也是进行多次的伯努利实验。

它指的是:在n次伯努利试验中,试验k次才得到第一次成功的机率。或者说,就是:前k-1次皆失败,第k次成功的概率。

几何分布的概率质量函数如下:
$$
P(X = k) = p(1-p)^{k - 1}, \quad k \in \{1, 2, 3, \cdots, \}
$$

几何分布的期望是$\frac{1}{p}$,方差是$\frac{(1-p)}{p^2}$

期望的推导过程:
$$
E(X) = \sum_{k=1}^{\infty}kp(1-p)^{k-1} = p\sum_{k=1}^{\infty}k(1-p)^{k-1}
$$

令 $q = 1 - p$,$S = \sum_{k=1}^{\infty}k(q)^{k-1}$,有:
\begin{align}
    S &= 1 + 2q + 3q^2 + 4q^3 + \cdots + kq^{k-1}\\
   qS &= q + 2q^2 + 3q^3 + 4q^4 + \cdots + kq^{k}\\
   S - qS &= 1 + q + q^2 + \cdots + q^{k-1} - kq^k \\
   S &= \frac{1-q^k}{(1-q)^2} - \frac{kq^k}{1-q} \\
     &= \frac{1}{(1-q)^2} \quad \text{这里因为} k \rightarrow \infty \\
     &= \frac{1}{p^2}
\end{align}
$$
E(X) = p\frac{1}{p^2} = \frac{1}{p}
$$

方差的推导过程:
$$
E(X^2) = \sum_{k=1}^{\infty}k^2p(1-p)^{k-1} = p\sum_{k=1}^{\infty}k^2(1-p)^{k-1}
$$

令 $q = 1 - p$,有
\begin{align}
    S &= \sum_{k=1}^{\infty}k^2q^{k-1} \\
      &= \sum_{k=1}^{\infty}(kq^k)' \quad' \text{表示求导} \\
      &= [\sum_{k=1}^{\infty}(kq^k)]' \\
      &= [(\frac{q}{1-q})^2]' \\
      &= \frac{(1-q)^2+2(1-q)q)}{(1-q)^4} \\
      &= \frac{2p-p^2}{p^4} \\
      &= \frac{2-p}{p^3}
\end{align}
所以:
$$
E(X^2) = \frac{2-p}{p^2}
$$
$$
D(X) = E(X^2) - [E(X)]^2 = \frac{2-p}{p^2} - (\frac{1}{p})^2 = \frac{1-p}{p^2}
$$

\subsection{多项分布}
进一步的,我们可以将二项分布扩展到多项(Multinomial)分布。

多项分布的结果有超过2种的更多种情况。例如:抛掷一枚骰子,其结果可能是1~6中的某个数值。

假设有$X_1$到$X_k$种结果,每种结果发生的概率是
$p_1$到$p_k$,多项分布的概率质量函数如下:
$$
P(X_1=n_1,...,X_k=n_k) = \frac{n!}{n_1! ... n_k !} p_1^{n_1} ... p_k^{n_{k}} , \; (\sum_{i=1}^k x_{i} = n)
$$
对于每个 $X_i$来说,其数学期望是$E(X_i) = np_i$,其方差是$Var(X_i) = np_i(1-p_i)$

\subsection{离散均匀分布}
特别地,当我们仅仅进行一次多项实验,并且多项的各项结果是等可能的,那么这个时候就得到的就是离散均匀(Discrete Uniform)分布。

其概率密度函数如下:
$$
P(X = x) = \frac{1}{N} \; (x= 1,...,N)
$$
例如,抛掷一枚均匀的骰子,出现6个数中任意一个的概率都是$\frac{1}{6}$。

离散均匀分布的期望值是$\frac{N+1}{2}$,方差是$\frac{(N+1)(N-1)}{12}$

期望的推导过程:
$$
E(X) = \sum_{k=1}^{N}\frac{1}{N}k = \frac{1}{N} \sum_{k=1}^{N}k = \frac{1}{N}\frac{(1+N)N}{2} = \frac{N+1}{2}
$$

方差的推导过程:
$$
E(X^2) = \sum_{k=1}^{N}\frac{1}{N}k^2 = \frac{1}{N} \sum_{k=1}^{N}k^2 = \frac{1}{N}\frac{N(N+1)(2N+1)}{6} = \frac{(N+1)(2N+1)}{6}
$$
$$
D(X) = E(X^2) - [E(X)]^2 = \frac{(N+1)(N-1)}{12}
$$

\subsection{上述几种分布的关系}
很显然,上面几种分布都与伯努利分布存在一定的关系,下面这幅图描述了它们之间的关系:
\begin{figure}[H]
  \centering
  \includegraphics[width=.8\textwidth]{fig/bernoulli_related.png} 
\end{figure}

\subsection{泊松分布}
泊松(Poisson)分布是另外一种很常见的概率分布,由法国数学家西莫恩·德尼·泊松(Siméon-Denis Poisson)在1838年时发表。

我们可以回想一下,生活中很多事情都以特定频率的反复发生的,例如:
\begin{itemize}
    \item 某个医院平均每天有100个新生儿;
    \item 某个客服号码每个小时会接到50个来电;
    \item 某一班公交每个小时会5次经过其中一个站点;
    \item 等等等等;
\end{itemize}

过去发生的平均频度我们是可以计算的,但是我们永远无法精确计算该事件下一次发生的时间点。

泊松分布描述的是:在已知过去发生频率的基础上,预测在接下来一段特定的时间内,该事件发生特定次数的概率。

泊松分布的概率质量函数如下:
$$
P(k) = \frac{e^{-\lambda}\lambda^{k}}{k!} \; ,\lambda \ge 0
$$

这里的$e$是一个常量,约等于2.71828。
$\lambda$是过去单位时间内发生的频率,$k$是预测发生的次数。

我们通过一个具体的例子就很容易理解泊松分布了。

以公交为例,假设我们知道过去它每个小时平均会5次经过其中一个站点($\lambda = 5$),那么接下来的一个小时,它经过的次数很可能是4~6次。不太可能是1次或者10次。我们可以根据概率质量函数,计算它接下来一个小时分别经过1次,4次,5次,10次的概率。

\begin{itemize}
\item 当 $k=1$ 时,$P(1) = \frac{e^{-5}5^1}{1!} \approx 0.034$
\item 当 $k=4$ 时,$P(4) = \frac{e^{-5}5^4}{4!} \approx 0.175$
\item 等等
\end{itemize}

泊松分布的期望值和方差都是$\lambda$。

期望的推导过程:
\begin{align}
E(X) &= \sum_{k=0}^{\infty}k\frac{e^{-\lambda}\lambda^{k}}{(k-1)!} \\
&= \lambda e^{-\lambda}\sum_{k=1}^{\infty}\frac{\lambda^{k-1}}{k!} \\
&= \lambda e^{-\lambda}e^{\lambda} \\
&= \lambda
\end{align}

这里也有一个变换(就是$e^x$的泰勒展开式):
$$
e^{\lambda} = \sum_{k=0}^{\infty}\frac{\lambda^k}{k!}
$$

方差的推导过程:
\begin{align}
E(X^2) &= \sum_{k=0}^{\infty}k^2\frac{e^{-\lambda}\lambda^{k}}{(k-1)!} \\
&= \sum_{k=1}^{\infty}k e^{-\lambda}\lambda \frac{\lambda^{k-1}}{(k-1)!} \\
&= \sum_{k=1}^{\infty}(k-1+1) \lambda e^{-\lambda} \frac{\lambda^{k-1}}{(k-1)!} \\
&= \sum_{k=1}^{\infty}(k-1) \lambda e^{-\lambda} \frac{\lambda^{k-1}}{(k-1)!} + \lambda e^{-\lambda} \sum_{k=1}^{\infty}\frac{\lambda^{k-1}}{(k-1)!} \\
&= \lambda^2 e^{-\lambda} \sum_{k=2}^{\infty}\frac{\lambda^{k-2}}{(k-2)!} + \lambda \\
&= \lambda^2 + \lambda
\end{align} 

$$
D(X) = \lambda^2
$$

\section{连续性分布\cite{Continuous_Variable_Expection_Variation}}
\subsection{高斯分布}
高斯(Gaussian)分布又称正态(Normal)分布。高斯分布的概率密度函数曲线呈钟形,因此人们又经常称之为钟形曲线。高斯分布的概率密度函数和累积分布函数如下图所示:
\begin{figure}[H]
  \centering
  \includegraphics[width=.8\textwidth]{fig/norm_dist.png} 
\end{figure}

从这个图中我们可以看出,对于高斯分布来说,随机变量处于中间的概率是比较大的,而其取非常大或者非常小的值的概率都很小。我们现实中人们的身高,体重,收入等特点都符合这个模型。

高斯分布的概率密度函数如下:
$$
f(x) = \frac{1}{\sqrt{2 \pi}\sigma}e^{-\frac{(x - \mu)^2}{2\sigma^2}}
$$

对于符合高斯分布的随机变量,我们也经常记做下面这样:
$$
X \sim N(\mu, \sigma^2)
$$

在这个函数中,$\mu$决定了高斯分布的中心位置,$\sigma$ 决定了钟型曲线的胖瘦程度。实际上,
$\mu$ 就是高斯分布的期望,而$\sigma^2$就是方差。

当 $\mu = 0, \sigma = 1$时,我们称之为标准正态分布。

期望的推导过程:
\begin{align}
E(X) &= \int_{-\infty}^{+\infty}xf(x)dx \\
&= \frac{1}{\sqrt{2 \pi}\sigma} \int_{-\infty}^{+\infty}xe^{-\frac{(x - \mu)^2}{2\sigma^2}}dx \quad \text{令} \frac{x-\mu}{\sigma} = t \\
&= \frac{1}{\sqrt{2 \pi}}\int_{-\infty}^{+\infty}(\sigma t+\mu)e^{-\frac{t^2}{2}}dt \\
&=  \frac{1}{\sqrt{2 \pi}}\int_{-\infty}^{+\infty}\sigma te^{-\frac{t^2}{2}}dt + \frac{1}{\sqrt{2 \pi}}\int_{-\infty}^{+\infty}{\mu}e^{-\frac{t^2}{2}}dt \quad \text{第一部分为奇函数} \\
&= \frac{1}{\sqrt{2 \pi}}{\mu}{\sqrt{2}}\int_{-\infty}^{+\infty}e^{-(\frac{t}{\sqrt{2}})^2}d{(\frac{t}{\sqrt{2}})} \\
&=  \frac{1}{\sqrt{2 \pi}}{\mu}{\sqrt{2}\sqrt{\pi}} \\
&= \mu
\end{align}

最后一步利用了积分公式:
$$
\int_{-\infty}^{+\infty}e^{-x^2}dx = \sqrt{\pi}
$$

方差的推导过程:
\begin{align}
E(X^2) &= \int_{-\infty}^{+\infty}x^2f(x)dx \\
&= \frac{1}{\sqrt{2 \pi}\sigma} \int_{-\infty}^{+\infty}x^2e^{-\frac{(x - \mu)^2}{2\sigma^2}}dx \quad \text{令} \frac{x-\mu}{\sigma} = t \\
&= \frac{1}{\sqrt{2 \pi}}\int_{-\infty}^{+\infty}(\sigma t+\mu)^2e^{-\frac{t^2}{2}}dt \\
&= \frac{1}{\sqrt{2 \pi}}\int_{-\infty}^{+\infty}({\sigma }^2t^2+2\mu\sigma t+{\mu}^2)e^{-\frac{t^2}{2}}dt \\
&= \cdots \\
&= {\sigma}^2 + {\mu}^2
\end{align}
$$
D(X) = E(X^2) - [E(X)]^2 = {\sigma}^2
$$

\subsection{均匀分布}
均匀(Uniform)分布要简单很多,它指的就是:随机变量在某个区间内,取任意一个值都是等可能的。

其概率密度函数如下:
$$
f(x) = \frac{1}{b-a} , a \le x \le b
$$
很显然,如果我们将这个函数画成图形,那就是两个区间之间的一个水平线。

均匀分布的期望值是$\frac{b+a}{2}$,方差是 $\frac{(b-a)^2}{12}$。

期望的推导过程:
$$
E(X) = \int_a^b\frac{x}{b-a}dx = \frac{1}{b-a}[\frac{1}{2}x^2]_a^b = \frac{b+a}{2}
$$

方差的推导过程:
$$
E(X^2) \int_a^b\frac{x^2}{b-a}dx = \frac{1}{b-a}[\frac{1}{3}x^3]_a^b = \frac{b^2+ab+a^2}{3}
$$
$$
D(X) = E(X^2) - [E(X)]^2 = \frac{(b+a)^2}{12}
$$

\subsection{指数分布}
指数(Exponential)分布是描述泊松过程中的事件之间的时间的概率分布,即事件以恒定平均速率连续且独立地发生的过程。举例来说,如果事件在每个时间点发生的概率相同,那么间隔时间的分布就近似于指数分布。

指数分布的概率密度函数如下:
$$
f(x) = \frac{1}{\lambda}e^{-\frac{x}{\lambda}} \; ,x \ge 0, \lambda > 0
$$

指数分布的期望是 $\lambda$,方差是 $\lambda^2$。

期望的推导过程:

(注:这里指数分布的概率密度函数定义为: $f(x) = {\lambda}e^{-{x}{\lambda}}$)

\begin{align}
E(X) &= \int_0^{+\infty}x{\lambda}e^{-{x}{\lambda}}dx \\
&= -\int_0^{+\infty}xd(e^{-\lambda x}) \\
&= -xe^{-\lambda x}\big|_0^{+\infty} + \int_0^{+\infty} e^{-\lambda x}dx \\
&= -\frac{1}{\lambda}e^{-\lambda x}\big|_0^{+\infty} \\
&= \frac{1}{\lambda}
\end{align}

方差的推导过程:
\begin{align}
E(X^2) &= \int_0^{+\infty}x^2{\lambda}e^{-{x}{\lambda}}dx \\
&= -\int_0^{+\infty}x^2d(e^{-\lambda x}) \\
&= -x^2e^{-\lambda x}\big|_0^{+\infty} + \int_0^{+\infty}2x e^{-\lambda x}dx \\
&= -x^2e^{-\lambda x}\big|_0^{+\infty} + \int_0^{+\infty}2x e^{-\lambda x}dx \\
&= \frac{2}{\lambda} E(X) \\
&= \frac{2}{\lambda^2}
\end{align}
$$
D(X) = E(X^2) - [E(X)]^2 = \frac{1}{\lambda^2}
$$


指数分布的概率密度函数和分布累积函数如下图:
\begin{figure}[H]
  \centering
  \includegraphics[width=.8\textwidth]{fig/expon_dist.png} 
\end{figure}

我们仍然是以前面提到的公交车为例。假设每个小时某班公交平均有5次会经过其中一个站点。看上图中我们特意在0.2,0.4和1.0这三个点上做的标记。从这个图形中我们可以看出,对于这班公交来说,12分钟(0.2小时)来车的概率是0.632,24分钟(0.4小时)来车的概率是0.865。 当等待的时间越接近一个小时,新的一班车就几乎肯定要来了。

\subsection{帕累托分布\cite{Think_Stats}}
帕累托分布的累积分布函数(CDF)为:
$$
CDF(x) = 1 - (\frac{x}{x_m})^{-\alpha}
$$

其中,参数 $x_m, \alpha$ 决定了分布的位置和形状。$x_m$ 是最小值。

TBD

\subsection{伽马分布}
TBD

\subsection{贝塔分布}
TBD

\subsection{卡方分布}
TBD

\subsection{柯西分布}
TBD

\subsection{概率密度和卷积\cite{Think_Stats}}
TBD

设两个随机变量 $X$ 和 $Y$ 的累积分布函数分别为 $CDF_X$ 和 $CDF_y$,并且 $Z = X + Y$,那么 $Z$ 服从什么分布呢?

可证明:两个随机变量的和的分布就等于两个概率密度的卷积,即:
$$
PDF_Z = PDF_Y * PDF_X
$$

\section{特征函数}
在概率论中,任何随机变量的特征函数完全定义了它的概率分布。

特征函数定义是:设$X$是实值随机变量,则对任意实数$t$,有:
$$
\phi(t) = Ee^{itX} = E(\cos{(tX)} - i\sin{(tX)}) = E(\cos{(tX)}) + iE(\sin{(tX)})
$$
称为随机变量 $X$ 的特征函数。

(特征函数是概率密度函数的连续傅里叶变换的共轭复数)

\subsection{特征函数为什么叫做特征函数\cite{Why_Called_Feature_Function}\cite{Understand_Feature_Function}}
因为一个分布的特征函数是与该分布密度互相决定的,也就是说,特征函数体现了并蕴含着该分布的全部特征。换言之,随机变量$X$和$Y$同分布,当且仅当它们有相同的特征函数(当然,同时也当且仅当它们有相同的分布密度)。

特征函数是随机变量的分布的不同表示形式。一般而言,对于随机变量 $X$ 的分布,大家习惯用概率密度函数来描述。比如说 $X$ 服从正态分布:$X \sim N(\mu,\sigma)$。

虽然概率密度函数理解起来很直观,但是确实随机变量$X$ 的分布还有另外的描述方式,比如特征函数。

\subsection{泰勒级数}
根据泰勒级数可知,两个函数$f(x)$ 和 $g(x)$ 的各阶导数相等的越多,那么这两个函数越相似。

也即是:
$$
\textbf{各阶导数都相等} \rightarrow f(x) = g(x)
$$

那么,随机变量分布的特征有吗?

\subsection{随机变量分布的特征}
随机变量的特征有如下:
\begin{itemize}
    \item 期望 $\mu$
    \item 方差 $\sigma^2$
    \item 偏态 $Skewness$
    \item 峰态 $Kurtosis$
    \item ……
\end{itemize}

这些特征具体是什么含义就不解释了,说来话长。不过这些特征都跟随机变量的“矩”有关系(什么是“矩”请参考“如何理解概率论中的矩?” )

如期望对应一阶矩,方差对应二阶矩,偏态对应三阶矩等。

直觉上可以有以下推论(其实还是有条件的,这里先忽略这些严格性,在实际应用中如下思考问题不大):

$$
\textbf{各阶矩相等} \rightarrow \textbf{各个特征都相等} \rightarrow \textbf{分布相同} 
$$

\subsection{特征函数}
随机变量 $X$ 的特征函数定义为:
$$
\phi X(t) = E[e^{itX}]
$$
为什么这么定义呢?首先,$e^{itX}$ 的泰勒级数为:
$$
e^{itX} = 1 + \frac{itX}{1!} - \frac{t^2X^2}{2!} + \cdots + \frac{(it)^nX^n}{n!}
$$

代入可以推出:
\begin{align}
\varphi X(t) &= E[e^{itX}] \\
&= E(1 + \frac{itX}{1!} - \frac{t^2X^2}{2!} + \cdots + \frac{(it)^nX^n}{n!}) \\
&= E(1) + E(\frac{itX}{1!}) - E(\frac{t^2X^2}{2!}) + \cdots + E(\frac{(it)^nX^n}{n!}) \\
&= 1 + \frac{it\overbrace{E[x]}^{\text{一阶矩}}}{1!} - \frac{t^2\overbrace{E[x^2]}^{\text{二阶矩}}}{2!} + \cdots +  \frac{(it)^n\overbrace{E[x^n]}^{\text{n阶矩}}}{n!} 
\end{align}
原来特征函数包含了分布函数的所有矩,也就是包含了分布函数的所有特征啊。所以我们可以进一步完善刚才的结论:
$$
\varphi X(t) \textbf{相等} \rightarrow \textbf{各阶矩相等} \rightarrow \textbf{各个特征都相等} \rightarrow \textbf{分布相同} 
$$

所以,特征函数其实是随机变量 $X$ 的分布的另外一种描述方式。

\subsection{特征函数和傅里叶变换的关系\cite{Understand_Feature_Function}}
TBD

\subsection{如何理解概率论中的“矩”?\cite{Understand_Moment_In_Prob}}
对比物理的力矩,你会发现,概率论中的“矩”真的是很有启发性的一个词。

\subsubsection{力矩}
大家应该都知道物理中的力矩,我这里也不展开说细节了,用一幅图来帮助大家回忆一下:
\begin{figure}[H]
  \centering
  \includegraphics[width=.5\textwidth]{fig/MomentInForce.png} 
\end{figure}
上图中,两边能保持平衡,只要满足下面的式子就可以了(很粗糙的式子,没把力作为向量来考虑):
$$
F_1D_1 = F_2D_2
$$
其中,$F_1D_1$ 和 $F_2D_2$ 都称为力矩。

\subsubsection{概率论中的“矩”}
首先举个彩票的例子:每一注2元,中奖概率分别为:5元:10\%,100元:0.5\%,500万:0.00001\%。

我们用概率来组装一把“秤”:
\begin{figure}[H]
  \centering
  \includegraphics[width=.5\textwidth]{fig/MomentInLottery.png} 
\end{figure}
把整张彩票都放上去称(秤上的刻度是随便画的,因为相差太悬殊,没有办法按照真是比例来画):
\begin{figure}[H]
  \centering
  \includegraphics[width=.5\textwidth]{fig/MomentInLottery2.png} 
\end{figure}
$1.5=5\times10\%+100\times0.5\%+5000000\times0.00001\%$ 这张彩票原来只值1.5元?血本无归啊!

\subsubsection{矩}
学过概率的都知道,我们上面计算的就是期望:$E(X) = \sum_ip_ix_i$,其实这就是“矩”,其中:
$$
p_i \quad \textbf{是秤上的刻度}
$$
$$
x_i \quad \textbf{是要称的重量}
$$
因为$x$是一次幂,所以也称为“一阶矩”。

再比如方差:$Var(X) = E[(X-\mu)^2] = \sum_ip_i(x_i-\mu)^2 $

其中的距离$(X-\mu)^2$也需要称量之后才能使用,所以方差也称为“二阶矩”。

“三阶矩”、“四阶矩”、“高阶矩”,各有用途,但是共同的特点就是称量之后才能使用。


%\printbibliography
\bibliography{../ref}
\bibliographystyle{IEEEtran}
\end{document}
