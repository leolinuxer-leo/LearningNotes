\documentclass[12pt]{article}
%\usepackage[utf8]{inputenc}
%\documentclass[UTF8]{ctexart}
%\usepackage[UTF8, heading = false, scheme = plain]{ctex}
\usepackage{geometry}
%geometry{a4paper,scale=0.9}
\geometry{a4paper,left=1cm,right=1cm,top=1cm,bottom=2cm}
\usepackage{amsfonts}
\usepackage{color}
\usepackage{url}
%\usepackage{biblatex}
\usepackage{amsmath}
\usepackage{amssymb}
\usepackage{latexsym}
\usepackage{listings}
\usepackage[usenames,dvipsnames]{xcolor}
\usepackage{cite}
%\addbibresource{ref.bib}
%\bibliography{ref.bib}
\usepackage{caption}
\usepackage{graphicx, subfig}
\usepackage{float}
%\usepackage[fontset=ubuntu]{ctex}
%\usepackage{fontspec}
\usepackage{xeCJK}
%\usepackage[colorlinks,
%anchorcolor=black,
%citecolor=black]{hyperref}
%\setmainfont{SimSun}
\usepackage[section]{placeins}
\usepackage{enumitem}
\usepackage{framed}
\usepackage[framemethod=TikZ]{mdframed}
\usepackage{indentfirst}
\usepackage{setspace}%使用间距宏包
\linespread{1.5}
\definecolor{mygreen}{rgb}{0,0.6,0}
\definecolor{mygray}{rgb}{0.5,0.5,0.5}
\definecolor{mybgray}{rgb}{0.95,0.95,0.95}
\definecolor{mymauve}{rgb}{0.58,0,0.82}
\lstset{
 backgroundcolor=\color{mybgray}, 
 basicstyle = \footnotesize,       
 breakatwhitespace = false,        
 breaklines = true,                 
 captionpos = b,                    
 commentstyle = \color{mygreen}\bfseries,
 extendedchars = false,             
 frame =shadowbox, 
 framerule=0.5pt,
 keepspaces=true,
 keywordstyle=\color{blue}\bfseries, % keyword style
 language = C++,                     % the language of code
 otherkeywords={string}, 
 numbers=left, 
 numbersep=5pt,
 numberstyle=\tiny\color{mygray},
 rulecolor=\color{black},         
 showspaces=false,  
 showstringspaces=false, 
 showtabs=false,    
 stepnumber=1,         
 stringstyle=\color{mymauve},        % string literal style
 tabsize=2,          
 title=\lstname                      
}

\title{数据结构和算法概述}
\author{leolinuxer}
%\date{June 2020}

\begin{document}
%\setlength{\parindent}{0pt}
\maketitle
\tableofcontents

\section{学习数据结构和算法的框架思维\cite{F_Algorithm_Mind_Framework}}
\subsection{数据结构的存储方式}
\textbf{数据结构的存储方式只有两种:数组(顺序存储)和链表(链式存储)}。

这句话怎么理解,不是还有散列表、栈、队列、堆、树、图等等各种数据结构吗?

我们分析问题,一定要有递归的思想,自顶向下,从抽象到具体。你上来就列出这么多,那些都属于「上层建筑」,而数组和链表才是「结构基础」。因为那些多样化的数据结构,究其源头,都是在链表或者数组上的特殊操作,API 不同而已。

比如说\textbf{「队列」、「栈」这两种数据结构既可以使用链表也可以使用数组实现}。用数组实现,就要处理扩容缩容的问题;用链表实现,没有这个问题,但需要更多的内存空间存储节点指针。

「图」的两种表示方法,\textbf{邻接表就是链表},\textbf{邻接矩阵就是二维数组}。邻接矩阵判断连通性迅速,并可以进行矩阵运算解决一些问题,但是如果图比较稀疏的话很耗费空间。邻接表比较节省空间,但是很多操作的效率上肯定比不过邻接矩阵。

\textbf{「散列表」就是通过散列函数把键映射到一个大数组里}。而且对于解决散列冲突的方法,拉链法需要链表特性,操作简单,但需要额外的空间存储指针;线性探查法就需要数组特性,以便连续寻址,不需要指针的存储空间,但操作稍微复杂些。

「树」,\textbf{用数组实现就是「堆」},因为「堆」是一个完全二叉树,用数组存储不需要节点指针,操作也比较简单;\textbf{用链表实现就是很常见的那种「树」},因为不一定是完全二叉树,所以不适合用数组存储。为此,在这种链表「树」结构之上,又衍生出各种巧妙的设计,比如二叉搜索树、AVL 树、红黑树、区间树、B 树等等,以应对不同的问题。

了解 Redis 数据库的朋友可能也知道,Redis 提供列表、字符串、集合等等几种常用数据结构,但是对于每种数据结构,底层的存储方式都至少有两种,以便于根据存储数据的实际情况使用合适的存储方式。

综上,数据结构种类很多,甚至你也可以发明自己的数据结构,但是底层存储无非数组或者链表,二者的优缺点如下:
\begin{itemize}
\setlength{\itemsep}{0pt}
\setlength{\parsep}{0pt}
\setlength{\parskip}{0pt}
    \item \textbf{数组}由于是紧凑连续存储,可以随机访问,通过索引快速找到对应元素,而且相对节约存储空间。但正因为连续存储,内存空间必须一次性分配够,所以说数组如果要扩容,需要重新分配一块更大的空间,再把数据全部复制过去,时间复杂度 O(N);而且你如果想在数组中间进行插入和删除,每次必须搬移后面的所有数据以保持连续,时间复杂度 O(N);
    \item \textbf{链表}因为元素不连续,而是靠指针指向下一个元素的位置,所以不存在数组的扩容问题;如果知道某一元素的前驱和后驱,操作指针即可删除该元素或者插入新元素,时间复杂度 O(1)。但是正因为存储空间不连续,你无法根据一个索引算出对应元素的地址,所以不能随机访问;而且由于每个元素必须存储指向前后元素位置的指针,会消耗相对更多的储存空间。
\end{itemize}

\subsection{数据结构的基本操作}
对于任何数据结构,其基本操作无非遍历 + 访问,再具体一点就是:增删查改。

数据结构种类很多,但它们存在的目的都是在不同的应用场景,尽可能高效地增删查改。话说这不就是数据结构的使命么?

如何遍历 + 访问?我们仍然从最高层来看,\textbf{各种数据结构的遍历 + 访问无非两种形式:线性的和非线性的}。

线性就是 for/while 迭代为代表。
\begin{framed}
理解:\textbf{线性遍历可以有正向(从前向后)和反向(从后向前)两种方式;对于一些问题,从前向后的话,可以有暴力解法;这时,可以思考从后往前是否有更加方案。}
\end{framed}

非线性就是递归为代表。再具体一步,无非以下几种框架:

数组遍历框架,典型的线性迭代结构:
\begin{lstlisting}
void traverse(int[] arr) {
    for (int i = 0; i < arr.length; i++) {
        // 迭代访问 arr[i]
    }
}
\end{lstlisting}

链表遍历框架,兼具迭代和递归结构:
\begin{lstlisting}
/* 基本的单链表节点 */
class ListNode {
    int val;
    ListNode next;
}

void traverse(ListNode head) {
    for (ListNode p = head; p != null; p = p.next) {
        // 迭代访问 p.val
    }
}

void traverse(ListNode head) {
    // 递归访问 head.val
    traverse(head.next)
}
\end{lstlisting}

二叉树遍历框架,典型的非线性递归遍历结构:
\begin{lstlisting}
/* 基本的二叉树节点 */
class TreeNode {
    int val;
    TreeNode left, right;
}

void traverse(TreeNode root) {
    traverse(root.left)
    traverse(root.right)
}
\end{lstlisting}

二叉树框架可以扩展为 N 叉树的遍历框架:
\begin{lstlisting}
/* 基本的 N 叉树节点 */
class TreeNode {
    int val;
    TreeNode[] children;
}

void traverse(TreeNode root) {
    for (TreeNode child : root.children)
        traverse(child)
}
\end{lstlisting}

\textbf{N 叉树的遍历又可以扩展为图的遍历,因为图就是好几 N 叉棵树的结合体}。你说图是可能出现环的?这个很好办,用个布尔数组 visited 做标记就行了,这里就不写代码了。

所谓框架,就是套路。不管增删查改,这些代码都是永远无法脱离的结构,你可以把这个结构作为大纲,根据具体问题在框架上添加代码就行了。

\subsection{算法刷题指南}
首先要明确的是,\textbf{数据结构是工具,算法是通过合适的工具解决特定问题的方法}。也就是说,学习算法之前,最起码得了解那些常用的数据结构,了解它们的特性和缺陷。

\textbf{先刷二叉树,先刷二叉树,先刷二叉树!}为什么要先刷二叉树呢,因为二叉树是最容易培养框架思维的,而且大部分算法技巧,本质上都是树的遍历问题。

刷二叉树看到题目没思路?根据很多读者的问题,其实大家不是没思路,只是没有理解我们说的「框架」是什么。不要小看这几行破代码,几乎所有二叉树的题目都是一套这个框架就出来了。
\begin{lstlisting}
void traverse(TreeNode root) {
    // 前序遍历
    traverse(root.left)
    // 中序遍历
    traverse(root.right)
    // 后序遍历
}
\end{lstlisting}

比如说我随便拿几道题的解法出来,不用管具体的代码逻辑,只要看看框架在其中是如何发挥作用的就行。

LeetCode 124 题,难度 Hard,让你求二叉树中最大路径和,主要代码如下:
\begin{lstlisting}
int ans = INT_MIN;
int oneSideMax(TreeNode* root) {
    if (root == nullptr) return 0;
    int left = max(0, oneSideMax(root->left));
    int right = max(0, oneSideMax(root->right));
    ans = max(ans, left + right + root->val);
    return max(left, right) + root->val;
}
\end{lstlisting}
你看,这就是个后序遍历嘛。

LeetCode 105 题,难度 Medium,让你根据前序遍历和中序遍历的结果还原一棵二叉树,很经典的问题吧,主要代码如下:
\begin{lstlisting}
TreeNode buildTree(int[] preorder, int preStart, int preEnd, 
    int[] inorder, int inStart, int inEnd, Map<Integer, Integer> inMap) {

    if(preStart > preEnd || inStart > inEnd) return null;

    TreeNode root = new TreeNode(preorder[preStart]);
    int inRoot = inMap.get(root.val);
    int numsLeft = inRoot - inStart;

    root.left = buildTree(preorder, preStart + 1, preStart + numsLeft, 
                          inorder, inStart, inRoot - 1, inMap);
    root.right = buildTree(preorder, preStart + numsLeft + 1, preEnd, 
                          inorder, inRoot + 1, inEnd, inMap);
    return root;
}
\end{lstlisting}
不要看这个函数的参数很多,只是为了控制数组索引而已,本质上该算法也就是一个前序遍历。

LeetCode 99 题,难度 Hard,恢复一棵 BST,主要代码如下:
\begin{lstlisting}
void traverse(TreeNode* node) {
    if (!node) return;
    traverse(node->left);
    if (node->val < prev->val) {
        s = (s == NULL) ? prev : s;
        t = node;
    }
    prev = node;
    traverse(node->right);
}
\end{lstlisting}
这不就是个中序遍历嘛,对于一棵 BST 中序遍历意味着什么,应该不需要解释了吧。

对于一个理解二叉树的人来说,刷一道二叉树的题目花不了多长时间。那么如果你对刷题无从下手或者有畏惧心理,不妨从二叉树下手,前 10 道也许有点难受;结合框架再做 20 道,也许你就有点自己的理解了;刷完整个专题,再去做什么回溯动规分治专题,你就会发现\textbf{只要涉及递归的问题,都是树的问题}。其实很多动态规划问题就是在遍历一棵树,你如果对树的遍历操作烂熟于心,起码知道怎么把思路转化成代码,也知道如何提取别人解法的核心思路。

%\printbibliography
\bibliography{../ref}
\bibliographystyle{IEEEtran}
\end{document}