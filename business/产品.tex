\documentclass[12pt]{article}
%\usepackage[utf8]{inputenc}
%\documentclass[UTF8]{ctexart}
%\usepackage[UTF8, heading = false, scheme = plain]{ctex}
\usepackage{geometry}
%geometry{a4paper,scale=0.9}
\geometry{a4paper,left=1cm,right=1cm,top=1cm,bottom=2cm}
\usepackage{amsfonts}
\usepackage{color}
\usepackage{url}
%\usepackage{biblatex}
\usepackage{amsmath}
\usepackage{amssymb}
\usepackage{latexsym}
\usepackage{cite}
%\addbibresource{ref.bib}
%\bibliography{ref.bib}
\usepackage{caption}
\usepackage{graphicx, subfig}
\usepackage{float}
%\usepackage[fontset=ubuntu]{ctex}
%\usepackage{fontspec}
\usepackage{xeCJK}
%\usepackage[colorlinks,
%anchorcolor=black,
%citecolor=black]{hyperref}
%\setmainfont{SimSun}
\usepackage[section]{placeins}
\usepackage{enumitem}
\usepackage{framed}
\usepackage[framemethod=TikZ]{mdframed}
\usepackage{indentfirst}
\usepackage{setspace}%使用间距宏包
\linespread{1.5}
%\title{预备知识}
%\author{leolinuxer }
%\date{June 2020}

\title{产品}
%\author{leolinuxer }
%\date{June 2020}

\begin{document}
%\setlength{\parindent}{0pt}
\maketitle
\tableofcontents

参考书籍:《俞军-产品方法论》

\section{主要观点}
用户价值 = 新体验 - 旧体验 - 替换成本

产品是约束条件下的效用组合

企业以产品为媒介,与用户进行价值交换;产品经理要能在实践中理解用户模型和交易模型,设计产品促成更多交易,以创造有利可图的用户价值。

产品是由人加工,有用户,且可以被交易的商品或服务。因此,产品需要经过需求、生产、销售三个环节。

产品经理的工作包括:需求(定义产品)、生产(产品设计和交付)、销售(和用户完成交易)、协调。

人类的知识体系:
\begin{itemize}
\setlength{\itemsep}{0pt}
\setlength{\parsep}{0pt}
\setlength{\parskip}{0pt}
    \item 自然科学:研究对象是自然界存在的物质,可证伪,可重复验证,所发现的规律一般具有普适性,如物理、化学、天文等;
    \item 形式科学:研究对象在真实自然界不存在,是人类建构的抽象规则,不涉及经验,不可能证伪,永远正确,如数学、逻辑、概率、编程语言等;
    \item 人文科学:研究对象是人的主观世界和精神活动,无法区分对错,无法证实,不适用于证伪,如宗教、哲学、诗歌、文学、美术、音乐、影视等;
    \item 社会科学:研究对象是人作为社会中的存在,或人与人组成的社会,如社会学、政治学、经济学、社会心理学等;社会科学与自然科学的最大区别是研究对象不可重复(人类意识多样,决定人类行为的动机和预期有相互博弈性和不确定性,社会永恒变化),虽然能应用自然科学的研究方法(观察、分析、控制实验等),但是必须以大量的经验和数据为依据,且控制实验事实上无法真正重复验证,社会科学的研究结论就总有不确定性,无法做到自然科学的高确定性。
\end{itemize}

交易模型:是指以交易为基本单位来研究产品,目标是建立可持续交易的互惠模型。搜索、社交、阅读等信息类产品是货币价格为零的特殊模型。研究交易模型,除了理解产品链上各方的价值判断和复杂关系,主要研究的就是交易费用。

产品是价值交换的媒介。企业以产品为媒介与用户进行价值交换。产品的关键不是其表面的功能、材料、属性,而是其深层的价值。企业与用户交换的不是产品,而是价值。

产品每增加一个属性,调整一个功能,都是在调整效用组合。

好产品的三个属性:对用户有效用,对企业有收益,可持续。

用户不是自然人,而是需求的集合。

用户的五个属性:异质性、情境性、可塑性、自利性、有限理性。
\begin{itemize}
\setlength{\itemsep}{0pt}
\setlength{\parsep}{0pt}
\setlength{\parskip}{0pt}
    \item 异质性:用户的特点千差万别;
    \item 情境性:用户的行为受情景的影响;
    \item 可塑性:用户是可变的;
    \item 自利性:用户追求个人总效用最大化;
    \item 有限理性:用户虽然追求理性,但他的能力是有限的,其判断经常出错,也经常被骗,所以只能做到有限的程度;
\end{itemize}

幸福 = 效用 - 欲望

欲望具有无限性和约束性。无限性是指人的所有行为,都是在约束条件下追求价值最大化,即欲望满足最大化,包括欲望种类的无限性和欲望满足程度的无限性。约束性是指存在各种约束条件,如金钱、时间、身体等。约束条件改变,人的行为改变。约束条件组合多变,欲望组合千人千面。

效用最大的特点是主观性。

用户感知到的价值才是价值。用户从产品获得和感知的,只是一组效用,且对这些效用的评估有个体差异和情景差异。无论客观上的产品有怎样的物理属性、设计属性、经营属性,用户感知到什么它就是什么,用户的主观认知是什么它就是什么。

用户价值具备认知依存、情境依存、经验反馈演化三个特点。
\begin{itemize}
\setlength{\itemsep}{0pt}
\setlength{\parsep}{0pt}
\setlength{\parskip}{0pt}
    \item 认知依存:用户的认知决定了他的偏好,比如喜欢喝酒还是咖啡等;
    \item 情境依存:有情境才有用户,脱离情境就没有用户;
    \item 经验反馈演化:用户价值是变化的;
\end{itemize}

海底捞认为翻台率高利润就大,所以他们通过优质服务让用户愿意排队,而不是涨价。因为涨价后,用户少了,不用排队,但是翻台率就低了。并且现在用户排队的时候都有手机,不是无所事事,所以排队的机会成本也降低了。

企业的本质:发现市场获利机会;生产效率高于市场;

发现市场获利机会的三种途径:
\begin{itemize}
\setlength{\itemsep}{0pt}
\setlength{\parsep}{0pt}
\setlength{\parskip}{0pt}
    \item 洞察:利用信息不对称获利;
    \item 试错;
    \item 偶然性:要尊重和敬畏这个世界的不确定性,产品结果必然有偶然性,人们不认可或者不了解的产品可能就会成长壮大;
\end{itemize}

阿罗信息悖论:在买方得到信息之前,他并不了解信息对他具有的价值;但是一旦他了解信息的价值,他事实上已经无成本地获得了这一信息。

大型市场的获利机会,一般有三种来源:市场环境和制度变化;关键新技术;长期关键因素——组织建设能力。

企业通过权威提高效率替代市场:企业建立科层制,通过权威来配置企业资源,组织实现更有效率的生产(提供物品或服务),效率必须高于市场。企业的运营不靠民主投票,企业也不是自由市场,企业的优势来自权威自上而下的更优决策。权威可以理解为掌握专业知识的员工。

影响组织效率的四要素:共同目标、共同理念、共同知识、运行机制。

\begin{itemize}
\setlength{\itemsep}{0pt}
\setlength{\parsep}{0pt}
\setlength{\parskip}{0pt}
    \item 使命:终极目标;
    \item 远景:阶段性中长期目标;
    \item 价值观:共同理念、取舍偏好;
\end{itemize}

《组织行为学》中提到的决定组织文化的三点:创始人的人格特性;高级管理者们的真实行为;员工的社会化,即日常向员工宣贯,以及通过制度和日常工作等让员工融入组织文化。

发展和生存是企业的两条腿。发展是创造用户价值的游戏;生存是不能有短板的游戏。

181家美国顶尖企业的管理者发出的联合声明,重新定义了一个公司运营的宗旨:股东利益不再是一个公司最重要的目标,公司的首要任务是创造一个更美好的社会。现代互联网企业已经在代行公共职能,包括通信、交易、教育、交通等。既然是行使公共职能,就要在一定程度上按照社会公器的标准与社会相处,找到新的平衡点,否则就难以持续。

产品的核心价值:使用价值和交换价值。使用价值即产品的各种效用,用户价值也是指使用价值;交换价值是一种使用价值同另一种使用价值相交换的量的比例或关系。交换价值是社会属性,一份产品的市场价格反应其交换价值,是市场供需博弈的结果。

没有使用价值的东西不能成为产品;但产品有使用价值,却不一定有交换价值。交换价值有三个属性:有效用、被认知、稀缺性,任一属性的变化都会严重影响交换价值。

用户购买或使用产品,是为了获取使用价值;但企业做产品却是为了交换价值。只有通过交换才能解决产品的使用价值和交换价值的矛盾。

创造价值的五大路径:劳动、分工、交易、新技术、制度。
\begin{itemize}
\setlength{\itemsep}{0pt}
\setlength{\parsep}{0pt}
\setlength{\parskip}{0pt}
    \item 
\end{itemize}

%\printbibliography
\bibliography{../ref}
\bibliographystyle{IEEEtran}
\end{document}
