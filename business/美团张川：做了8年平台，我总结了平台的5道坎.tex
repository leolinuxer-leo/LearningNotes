\documentclass[12pt]{article}
%\usepackage[utf8]{inputenc}
%\documentclass[UTF8]{ctexart}
%\usepackage[UTF8, heading = false, scheme = plain]{ctex}
\usepackage{geometry}
%geometry{a4paper,scale=0.9}
\geometry{a4paper,left=1cm,right=1cm,top=1cm,bottom=2cm}
\usepackage{amsfonts}
\usepackage{color}
\usepackage{url}
%\usepackage{biblatex}
\usepackage{amsmath}
\usepackage{amssymb}
\usepackage{latexsym}
\usepackage{cite}
%\addbibresource{ref.bib}
%\bibliography{ref.bib}
\usepackage{caption}
\usepackage{graphicx, subfig}
\usepackage{float}
%\usepackage[fontset=ubuntu]{ctex}
%\usepackage{fontspec}
\usepackage{xeCJK}
%\usepackage[colorlinks,
%anchorcolor=black,
%citecolor=black]{hyperref}
%\setmainfont{SimSun}
\usepackage[section]{placeins}
\usepackage{enumitem}
\usepackage{framed}
\usepackage[framemethod=TikZ]{mdframed}
\usepackage{indentfirst}
\usepackage{setspace}%使用间距宏包
\linespread{1.5}
%\title{预备知识}
%\author{leolinuxer }
%\date{June 2020}

\title{美团张川:做了8年平台,我总结了平台的5道坎}
%\author{leolinuxer }
%\date{June 2020}

\begin{document}
%\setlength{\parindent}{0pt}
\maketitle
\tableofcontents

\begin{framed}
文 | 美团到店事业群总裁  张川

\url{https://mp.weixin.qq.com/s/Egsznk8qffi_2ppTiMaSxg}

中国互联网20年,所有的大小巨头无一例外,都是平台。大行业不一定能产生大公司,但是大平台一定意味着大公司。

同样是平台,为什么交易平台比信息平台的价值和规模更大?同样是交易平台,为什么今天商品平台的规模领先生活服务平台?为什么中国律师网、中国好医生这种平台注定做不起来?

平台的商业逻辑,远不是一个万能的“流量入口”那么简单。

同时,做平台是有很多方法论的,而这些方法论是被无数的成功和无数的失败经验所证明过的。

比如,有的二手车平台,从一开始就做错了——1.他们拼命补贴用户,但买卖二手车是个低频需求,\textbf{低频需求靠广告,高频需求才靠补贴};2.他们面向买家打广告,但二手车市场供不应求,供给端改革才是核心。所以给买家打广告的人人车对比给卖家打广告、强调卖车赚钱的瓜子,后者才真正意识到了商业的本质。

平台是一个听起来简单,但理解起来难,做起来更难的事情。

美团点评到店事业群总裁张川基于自己多年的实战经验,对平台做了深入的思考和总结。

张川有近15年的互联网及IT行业经历,曾担任过58集团执行副总裁、百度联盟产品负责人。美团创始人王兴曾评价张川:“他在互联网产品技术领域、商业产品设计、商业体系建设等方面有非常丰富的成功经验。”

张川既做过低频业务,又做过高频业务,因此也有人称他是最了解平台的人之一。

以下是他给《财经》读者的分享:
\end{framed}

\section{第一关:动态不平衡才能形成真正的平台}

\textbf{平台遇到的第一关,是能否确定、并识别出平台的两端是极其不平衡的业务,并通过运营、产品,保持两端的不平衡,这样才能最终成就平台的价值}。

这部分主要思考的,是商业社会最常见的Two Sided Marketplace(双边平台),如淘宝、美团、滴滴都是典型的双边平台。

\textbf{动态不平衡,即这个市场的活跃度足够高,不会产生单个用户和单个服务提供者在一段时间内多次达成同一个交易的过程。}

例如:用户交易的时候很少在固定的时候固定的买一家店固定的商品,用户也很少同时同刻在同一地点打上同一个司机的同一辆车。

\textbf{平台的基础,是两端要形成动态不平衡。只有两端动态不平衡,平台才有生存的价值。}

这里面会有两个陷阱:表面看上去是动态不平衡,但实际上是平衡的。

\subsection{第一个陷阱是“初始不平衡,结尾平衡”}

\textbf{第一个陷阱是“初始不平衡,结尾平衡”}。比如家教、美容美发。初始状态是不平衡的,但你会经常用一个人的服务,因为服务过程主要是交流的过程,所以用户和服务者之间有很多的情感依赖。美容很多时候变成了情感诉说,自然就不会换人了。

当平衡态发生,人们就会开始私自成交,平台的作用就越来越不明显了。保姆、小时工、上门做饭、汽车维修都是这样。

如何走出这个陷阱、在上述产业中取得突破?

1.\textbf{标准化服务,把情感的因素降低到最低}。例如:新兴的理发行业——理发标准化,你找哪个理发师的结果都一样,这样就变成一个不平衡性质的平台了。

2.\textbf{拆细服务,并且不依赖于人}。例如:健身机构把健身的某些功能拆解出来,例如变成脊柱矫正锻炼,和私教没有关系。

比如说英语在线教育最大的创新,就是将外语培训拆解成很多标准的语言,所以今天你可以跟这个老师上一节口语课,明天可以跟那个老师上一节口语课。从而变成标准化的系列服务,有效地将单次服务变成多次服务。

\subsection{第二个陷阱是“平台专家陷阱”}

\textbf{第二个陷阱是“平台专家陷阱”}。如果平台是以知识,特别是独有性知识为前提的平台,容易掉入这个陷阱。典型的就是律师、医生、老师。这类平台用户选择成本很高,需求会集中到专家身上。当平台养出好的服务者以后,他可能就离开平台。

\textbf{专家型的陷阱会让很多创业者一开始觉得平台做得还不错,但是很快就进入平台的停滞期,随着用户和专家的流失,平台进入衰退期。}

如何破解专家陷阱?

要挑出专家和普通服务者差距不大的服务,例如,医疗的体检,律师中的交通违章。这些是对专家化服务标准化的探索方向。

\subsection{关于外卖市场的不平衡性}

最后谈谈,我过去判断错的一点,我曾经认为外卖是个初始不平衡,但是结果平衡的业务,因为一个人需要7-8家餐厅就好了,很容易平衡。

但后来发现不是,人的口味是一定要换的,所以人和餐厅之间是个动态不平衡,而且平台提供支付,配送等服务的时候,平台的价值得到增值;同时,外卖平台大大扩展了选择性,一般提供上百家的餐厅进行选择。所以,外卖平台成为很大的平台。

大家可思考一下:城际货运市场是不是一个动态不平衡的市场? 本地货运市场是不是一个动态不平衡的市场?

\section{第二关:标准化决定平台大小}
动态不平衡决定平台是否能存在。而平台内提供的东西是否是标准化的,决定了平台的大小。

\textbf{商品的标准化是容易的,但是生活服务的标准化都是不容易的,只有相对的标准化。所以目前看实物电商规模比生活服务电商更大}。当然随着生活服务逐步标准化,未来的前景很巨大。

本节讨论的是服务的标准化,不是信息的标准化。标准化影响的是服务的交易过程,如果是标准化的服务,交易流程变得非常简单。如果非标,一般很难产生交易。

\subsection{鲜花和水果谁更标准化?}

鲜花因为只有观赏的作用,在荷兰市场上,根据大小颜色,鲜花是有标准化定义的,所以鲜花市场的交易变得非常简单。水果因为有了味道,只看表面很难确立标准,所以水果的标准化就比鲜花更加困难。

\subsection{打车和搬家谁更加标准化?}

打车更多是安全到达目的地,主要内容是运,所以相对是标准化的;但是搬家本身是个“搬”和“运”的组合,搬是非常不标准化的,例如楼层,是否需要拆装家具,是否有冰箱和钢琴,这些都是非常不标准化的,\textbf{这些不标准化就影响了交易平台的形成}。

所以,运的行业出现了滴滴。搬的行业中目前还没有相关的平台成立。

\subsection{如何判断服务是否可以标准化?}

\textbf{我的方法论是——服务的体验可以一致化,客户的评价可以标准化,而不是依赖于服务的复杂度,更不是依赖于复杂服务的时长和难度。}

比如,法律上企业的注册和企业的商标申请,是一个非常复杂、耗时的服务,但其实就是标准化的服务——其评价的标准化在于——申请下来,评价好;申请不下来,评价不好,用户对它的评价标准是一致的。

再比如,保姆是个非常不标准化的服务,任何平台目前的解决方式都很难标准化,因为大家的需求不一样、喜好不一样,每个人的认知和评价不一致。

\subsection{如何将不标准化的服务变成标准化的服务?或者在不标准化的服务上形成平台呢?}
\textbf{第一个方案是,想办法把复合型的服务拆解开来,变成一些可以标准化的分步骤。}

例如,健身是个非标的服务,但是可以将颈椎治理的部分标准化,不同的健身教练都可以上。

\textbf{第二个方案是,不做交易的平台了,只做信息的平台。}

例如,保姆、家教可以做一个信息平台。

还有一种是包含复杂决策的生活服务类别,例如装修、婚庆。前端的试看、讲解可以标准化——信息和知识永远是可以标准化复制的。

还有一种是包含复杂决策的生活服务类别,例如装修、婚庆。\textbf{前端的试看、讲解可以标准化——信息和知识永远是可以标准化复制的}。

例如,驾考的前置是学习,驾校一点通和驾考宝典做的不错,Houzz是装修的前置知识标准化,结婚的前置有很多的工具,也是可以标准化的和平台化的。但是注意,这里标准化的还是信息和知识,不是交易本身。

二手房市场是一个动态不平衡的市场,但房产交易类商品太不标准化了,它导致要做就需要平台做N种的工作才能做到,所以,房产电商能否成功,要看能不能把它标准化到极致。

标准化里面,还有一个反向标准化的过程,就是C2B的过程,这里面的代表thumbtack,但是这个过程最后证明也不是特别成功。总而言之,在把一个非标服务标准化的过程中,过去已经做了很多的探索。成功的不多,失败的不少。

\section{第三关:高频打低频是误解}
一个平台是做高频业务,还是做低频业务,同样决定了平台价值的大小。

过去有一种说法——高频打低频。这是一个误解,高频其实挺难打到低频的。我觉得\textbf{高频之后打低频全部是靠规模效应打掉、靠资本力量打掉,它不是靠高频的用户慢慢迁移到了低频}。

我的观察有几点:

\subsection{第一,高频不能带动低频,或者说高频带动低频不太明显。}
如果是低频服务,用户一年使用一次,平台难以给到流量、品牌资源支持,所以靠自然流量的溢出,很难形成绝对的优势。

准确的做法是,高频带动中频,形成巨大的用户平台,然后优化低频体验。中频服务,例如酒店、KTV、电影票可以被高频服务带动。3C可以被高频服务中的服装带动。

\subsection{第二,高频服务靠补贴,低频服务靠广告。}
之前的二手车广告大战,还在房产、招聘的领域出现过。外卖、单车、打车等高频服务都是靠补贴做起来的。

低频的服务补贴起不到作用,以二手车举例,给了用户3000元补贴,卖车可能是几年后,品牌认知已经完全遗忘了,所以一个在二手车行业做补贴的公司是最先倒下来的一家。

另外,低频服务靠广告建立起来的品牌壁垒,比大多数人认为的都要浅。因为低频服务一般的广告周期都在3-5年,也只影响了用户感觉,所以有时候另外一家新进入者在5年后做广告宣传,也是非常有力量的,甚至比第一家做广告的效果还要好。这个是我自己的预言,也是一个个人认知。所以,我认为一个低频服务要进行至少10年的广告轰炸才管用。

\subsection{第三,多个低频可以聚集成高频。}
平台很难从高频带动低频,但是可以做众多的低频业务聚合起来,形成用户认知。聚沙成塔,蚂蚁雄兵,也可以形成一个有效的平台。

综合第一和第三的观点,就是有一个平台可以高频开始带动中频,然后不断开拓低频,是可以不断形成更大的平台的。淘宝不断威胁京东也是这样的原因。

多个低频聚成高频,另一个好处,就是平台能不能帮助用户和商家博弈。

这里有一个认知,\textbf{低频生意的商家不是做回头客生意的,所以从本质上而言就是要在一个用户身上赚到足够多的利润,同时一个低频用户对服务也是不太了解的,所以这里有很大的信息不对称}。

平台代表用户和商家博弈的好处就是,平台将多个低频用户转化为高频,可以更好地转变低频的服务和口碑。

这里面还存在一个平台作恶的悖论,例如,一些婚庆和装修的平台,因为管理的问题,平台利益和销售利益不一致,销售组织利用平台帮助供应商骗人,这让平台反而有负面的影响,这个就要有强大的管理能力。

\subsection{第四,低频服务很难出现好的产品经理。时间跨度长,产品不好进行优化。}
做租房子的,你是一年以后才知道体验是好还是不好;如果是做装修的,就是5年以后,汽车是10年以后。所以,在低频服务中,产品经理和业务如果可以整合,就还会有不错的结果。

在低频服务中,还有一种特殊的存在,就是总体上看低频,但是在目标人群是高频,例如:代驾,但是这些市场相对而言非常小。

这个市场尽量不要碰,因为这种业务的护城河还是太浅,很容易被别人侵占。如果是个很专业的市场,也许有很深的壁垒,但是市场规模不大。

\section{第四关:供给端的效率高,平台价值大}
\textbf{经济学原理中有一句话:短期看需求,长期看供给。}

平台的供给主要看两个方向:供给是不是可以大批量供给,并且接近于无限供给;是不是平台提高了供给端的效率,让供给端能赚到钱。

这两点必须有一点成立,交易平台才容易做大。

\subsection{首先我们要理解,供给的快和慢。}
一个好的平台,需要供应商的数量不断提升,例如,UBER和AIRBNB,司机和汽车在现在的社会中几乎等于无限供给,而GPS导航使熟练司机的数量大幅度增加。房子也是这样的,所以AIRBNB也发展很快。

或者是商品和服务结合的产品,由于供应者的时间稀缺性,一个按摩师傅一天只能服务有限的人群,而且好的按摩师傅需要很长的时间进行培养,一旦脱离平台,新增加的难度非常大,供给速度和供给量非常影响平台的速度。

未来的发展,要从机器替代人的过程思考,例如家务机器人,自动按摩椅等等,才能真正的解决供给量的问题。

\subsection{第二个是要理解,"供大于求,供不应求"的问题。}
在分类信息世界中,信息的供给永远是一个问题,分类信息和黄页不同,分类信息标准的是一货一品,一货一价,成交会影响其他人,例如二手房和二手车,所以分类信息叫做Classified Ads。黄页是标准的合适供给满足多个需求,每一次成交对他人没有影响,所以叫做Yellow Page。

很多人容易搞混两个服务。例如,招聘信息更加偏向于分类信息,而不是 Yellow Page。

\textbf{在分类信息领域内,尽量解决供给的问题}:

二手车,永远是供不应求的问题,只要有合适的车,一定会有买家。所以一个二手车网站最重要的任务是让更多的人在这个网站上销售汽车,并不是为浏览者服务的。

瓜子理解了这个业务,在广告上打出了卖家多卖钱,这个反应了其认识了生意的本质,而其他二手车平台实际上还是认为这是个需求端的生意,主要从用户端说诉求,这是不对的。

\textbf{二手房和租房,在平台看来,也是首先要关注二手房的售卖和出租者}。举例,二手房很多,实际上在一个小区内也就是3-5套供给,有5套供给的那家,就是要比3套供给的用户体验好很多。

从这个方向而言,房产平台的广告应该思考如何提高供给,要确保每个城市的房源数量和房源质量都优于竞对,才可以大规模地有广告效果,否则效率就是会很低。

\textbf{要关注,二手房是不是即将进入一个供大于求的时代,这样平台的价值就不太一样了。}

对于酒店和KTV而言,也是供给密度要够,才能满足用户需求,但是酒店肯定处于供大于求的时代,所以平台的抽佣比例相对而言就是会高一些。

有些市场的供给平衡是相对简单的,例如搬家公司,全北京和上海有300家搬家公司就够用了,保持多样性就够了,这个的供给就是找到这300家合适的搬家公司,不断优化并且淘汰,这个供给问题解决起来相对而言简单,但是如何维护合理的供给和提供优胜劣汰的生存环境,反而是越来越重要的。


\subsection{第三个,理解没有稳定供给的市场,不会是一个巨大无比的市场。}
例如二手的市场在我看起来可能不是一个巨大无比的市场。二手市场中最好的是二手车、二手房,在中国这样人口居多市场中,容易成立。但是单纯的二手货,因供给的不稳定和供给体验的不完善,抑制了需求的产生,在有淘宝和拼多多这样稳定便宜供给的竞争下,可能不是一个大规模的需求。

在2015年出现了很多的上门服务,上门按摩、上门美发、上门汽车,大家陷入上门服务的体验优化的过程中,其实没有提升供给的效率,一个按摩师傅在店里面可以做10单,在上门的过程中可能就只能做4单,如果单价是一样的,其实效率是降低的,只是由于平台的补贴形成了虚假的繁荣。

外卖由于骑手的参与,其实外卖平台提供的效率远远高于单个门店送餐的效率,即使原来补贴也很多,平台没有补贴后需求依然很多。

\section{第五关:想清楚自己的商业模式,剃须刀还是电冰箱?}
平台到了最后构造商业模式的时候,一定要想清楚是做是做剃须刀(LTV,生命周期总价值),还是电冰箱(CAC,用户获取成本)的生意。

这两个生意很多具体操作是交融的,但本质不一样。剃须刀的生意,就是看中一个客户的长期价值,第一次生意不追求赚钱,甚至是亏钱的,但是依靠长期卖刀片的生意,把利润做高做大。打印机生意同理。

电冰箱的生意,就是一定要在本次交易中覆盖用户获得成本,生意的公式就是本次收入-获客成本,不要期待未来还能依靠给冰箱卖鸡蛋能够从单个客户上获得更多的收入。

无论是什么生意,都需要关注NPS(净推荐值),客户推荐的概率,对于低频和高频的服务都特别的重要,因为只有这个名词在低频业务里面降低CAC(获客成本),在高频业务中提升LTV(生命周期总价值)。

例如,在square的发展过程中,产品做得好,客户之间推荐概率非常高,有40\%的客户是推荐而来的,大大降低了Square公司的获客成本,这是Square公司的生存之道。

但是,不是所有的生意都完全符合NPS的定义,有些生意是零和游戏的生意。例如广告,百度公司的效果好,但是商户并不会推荐给同行,因为同行使用了以后,他的损失就会增大。所以有些生意的本质上不是追求NPS。

Square公司目前的关键点就在从CAC到LTV的转化过程中。

从客观的角度看一下百度的生意和58的生意,百度的生意的本质在于LTV,因为百度发展每个客户的成本都非常高,第一次成交基本上的收入是低于销售的成本的,所以在于不断续费,提高服务,提升LTV。

58的生意一开始在于降低CAC,因为多数商家都是从58发帖的,所以如何一开始让更多的商家使用58就变得非常重要,因为是小商家,也不能设计复杂的产品,所以开始阶段全部是降低CAC成本。

\textbf{当年我对58生意最大的改造就是在58增加了LTV成分,融合了点击付费和置顶,这就形成了整体的58的收入增加}。未来,58的商业产品的关键点,是寻找可以增加LTV的部分,而不是寻求不断的降低CAC;百度的商业关键点是找到可以降低CAC的方法,使自己的客户数增长到几百万。

以上的5个思考是我觉得生活服务业非常重要的思考逻辑和方法论,这些方法论主要来源于实践和观察,并不是来源于很多的对理论的研究,也欢迎大家跟我持续交流,纠正思考。

%\printbibliography
\bibliography{../ref}
\bibliographystyle{IEEEtran}
\end{document}
