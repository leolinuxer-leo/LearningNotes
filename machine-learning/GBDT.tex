\documentclass[12pt]{article}
%\usepackage[utf8]{inputenc}
%\documentclass[UTF8]{ctexart}
%\usepackage[UTF8, heading = false, scheme = plain]{ctex}
\usepackage{geometry}
%geometry{a4paper,scale=0.9}
\geometry{a4paper,left=1cm,right=1cm,top=1cm,bottom=2cm}
\usepackage{amsfonts}
\usepackage{color}
\usepackage{url}
%\usepackage{biblatex}
\usepackage{amsmath}
\usepackage{amssymb}
\usepackage{latexsym}
\usepackage{cite}
%\addbibresource{ref.bib}
%\bibliography{ref.bib}
\usepackage{caption}
\usepackage{graphicx, subfig}
\usepackage{float}
%\usepackage[fontset=ubuntu]{ctex}
%\usepackage{fontspec}
\usepackage{xeCJK}
%\usepackage[colorlinks,
%anchorcolor=black,
%citecolor=black]{hyperref}
%\setmainfont{SimSun}
\usepackage[section]{placeins}
\usepackage{enumitem}
\usepackage{framed}
\usepackage[framemethod=TikZ]{mdframed}
\usepackage{indentfirst}
\usepackage{setspace}%使用间距宏包
\linespread{1.5}
%\title{预备知识}
%\author{leolinuxer }
%\date{June 2020}


\title{GBDT的原理\cite{Zhihu_Understanding_GBDT}\cite{Zhihu_Understanding_GBDT_2}}
%\author{leolinuxer }
%\date{June 2020}

\begin{document}
\maketitle

\section{如何在不改变原有模型的结构上提升模型的拟合能力}

假设现在你有样本集$(x_1,y_1),(x_2,y_2),\cdots,(x_n,y_n)$,然后你用一个模型,如$F(x)$去拟合这些数据,使得这批样本的平方损失函数(即$\frac{1}{2}\sum_0^n(y_i-F(x_i))^2$)最小。但是你发现虽然模型的拟合效果很好,但仍然有一些差距,比如预测值 $F(x_1)=0.8$,而真实值$y_1=0.9$,另外你不允许更改原来模型$F(x)$的参数,那么你有什么办法进一步来提高模型的拟合能力呢。

既然不能更改原来模型的参数,那么意味着必须在原来模型的基础之上做改善,那么直观的做法就是建立一个新的模型 $f(x)$ 来拟合 $F(x)$ 未完全拟合真实样本的残差,即$y-F(x)$。所以新模型需要拟合的样本集就变成了:$(x_1,y_1-F(x_1)),(x_2,y_2-F(x_2)),\cdots,(x_n,y_n-F(x_n))$

\section{基于残差的 GBDT}
在第一部分,$y_i-F(x_i)$被称为残差,这一部分也就是前一模型($F(x_i)$)未能完全拟合的部分,所以交给新的模型来完成。

我们知道 GBDT 的全称是Gradient Boosting Decision Tree,其中 Gradient 被称为梯度,更一般的理解,可以认为是一阶导,那么这里的残差与梯度是什么关系呢。在第一部分,我们提到了一个叫做平方损失函数的东西,具体形式可以写成$\frac{1}{2}\sum_0^n(y_i-F(x_i))^2$,熟悉其他算法的原理应该知道,\textbf{这个损失函数主要针对回归类型的问题,分类则是用熵值类的损失函数}。具体到平方损失函数的式子,你可能已经发现它的一阶导其实就是残差的形式,所以基于残差的 GBDT 是一种特殊的 GBDT 模型,它的损失函数是平方损失函数,常用来处理回归类的问题。具体形式可以如下表示:

\textbf{损失函数:} $L(y,F(x)) =\frac{1}{2}(y_i-F(x))^2 $

\textbf{我们希望最小化:} $J=\frac{1}{2}\sum_0^n(y_i-F(x_i))^2$

损失函数的一阶导:
$$\frac{\partial J}{\partial F(x_i)} = \frac{\partial \sum_iL(y_i,F(x_i))}{\partial F(x_i)}=\frac{\partial L(y_i,F(x_i))}{\partial F(x_i)} = F(x_i)-y_i$$

正好残差就是负梯度:
$$
y_i-F(x_i) = -\frac{\partial J}{\partial F(x_i)} 
$$

\section{为什么基于残差的 GBDT 不是一个好的选择}

基于残差的gbdt在解决回归问题上不算是一个好的选择,一个比较明显的缺点就是对异常值过于敏感。所以一般回归类的损失函数会用绝对损失或者 Huber 损失函数来代替平方损失函数:

\begin{itemize}[itemindent=2em]
    \item 绝对值 (absolute loss): 
    $L(y,F) = |y-F|$
    
    \item Huber损失 (huber loss): 
    $L(y,F) = \begin{cases}
\frac{1}{2}(y-F)^2 & |y-F| <= \delta\\
\delta|y-f(x)|-\frac{1}{2}\delta^2 & |y-F| > \delta\\
\end{cases}$
\end{itemize}

\section{Boosting的加法模型}
如前面所述,GBDT 模型可以认为是是由 k 个基模型组成的一个加法运算式:
$$
\hat{y_i} = \sum_{k=1}^K{f_k(x_i)}, f_k \in F
$$

其中 F 是指所有基模型组成的函数空间。

那么一般化的损失函数是预测值 $\hat{y}$ 与 真实值 $y$ 之间的关系,如我们前面的平方损失函数,那么对于n个样本来说,则可以写成:
$$
L = \sum_{i=1}^n{l(y_i,\hat{y_i})}
$$

更一般的,我们知道一个好的模型,在偏差和方差上有一个较好的平衡,而算法的损失函数正是代表了模型的偏差面,最小化损失函数,就相当于最小化模型的偏差,但同时我们也需要兼顾模型的方差,所以目标函数还包括抑制模型复杂度的正则项,因此目标函数可以写成:
$$
Obj = \sum_{i=1}^n{l(y_i,\hat{y_i})} + \sum_{k=1}^{K}\Omega(f_k)
$$

其中 $\Omega(f_k)$ 代表了基模型的复杂度,若基模型是树模型,则树的深度、叶子节点数等指标可以反应树的复杂程度。

对于Boosting来说,它采用的是前向优化算法,即从前往后,逐渐建立基模型来优化逼近目标函数,具体过程如下:

\begin{eqnarray*}
 && \hat{y}_i^0 = 0 \\
 && \hat{y}_i^1 = f_1(x_i)= \hat{y}_i^0 + f_1(x_i)\\
 && \hat{y}_i^2 = f_1(x_i) + f_2(x_i) = \hat{y}_i^1 + f_2(x_i) \\
 && \cdots \\
 && \hat{y}_i^t = \sum_{k=1}^tf_k(x_i) = \hat{y}_i^{t-1} + f_t(x_i)
\end{eqnarray*}

那么,在每一步,如何学习一个新的模型呢,答案的关键还是在于 GBDT 的目标函数上,即新模型的加入总是以优化目标函数为目的的。

我们以第t步的模型拟合为例,在这一步,模型对第 $i$ 个样本 $x_i$ 的预测为:
$$
\hat{y}_i^t = \hat{y}_i^{t-1} + f_t(x_i)
$$

其中 $f_t(x_i)$ 就是我们这次需要加入的新模型,即需要拟合的模型,此时,目标函数就可以写成:
$$
Obj^{(t)} = \sum_{i=1}^n{l(y_i,\hat{y}_i^t)} + \sum_{i=i}^{t}\Omega(f_i) = \sum_{i=1}^n{l(y_i,\hat{y}_i^{t-1} + f_t(x_i))} + \Omega(f_t) + constant
$$

即此时最优化目标函数,就相当于求得了$f_t(x_i)$

\section{什么是 GBDT 的目标函数}
根据泰勒公式推导二阶导(GBDT是一阶导,xgboost是二阶导):
$$
f(x+\Delta x) \approx f(x) + f'(x)\Delta x + \frac{1}{2}f''(x)\Delta x^2
$$

\textbf{建立 $Obj$ 表达式和二阶泰勒展开的对应关系}:

\begin{itemize}[itemindent=2em]
    \item $l(y_i,\hat{y}_i^{t-1})$ 对应泰勒公式中的$f(x)$
    
    \item $\hat{y}_i^{t-1}$ 对应泰勒公式中的 $x$
    
    \item $f_t(x_i)$ 对应泰勒公式中的 $\Delta x$
    
    \item $l(y_i,\hat{y}_i^{t-1} + f_t(x_i))$ 对应泰勒公式中的 $f(x+\Delta x)$
\end{itemize}

那么,对 $l(y_i,\hat{y}_i^{t-1} + f_t(x_i))$ 进行二阶泰勒展开后,可以得到:
$$
Obj^{(t)} = \sum_{i=1}^n[l(y_i,\hat{y}_i^{t-1}) + g_if_t(x_i) + \frac{1}{2}h_if_t^2(x_i)] + \Omega (f_t) + constant
$$

其中,
\begin{itemize}[itemindent=2em]
    \item $g_i$ 是损失函数的一阶导,对应泰勒公式中的 $f'(x)$
    
    \item $h_i$ 是损失函数的二阶导,对应泰勒公式中的 $f''(x)$
\end{itemize}

注意是对 $\hat{y}_i^{t-1}$ 求导。以 平方损失函数为例:
\begin{eqnarray*}
    && \sum_{i=1}^n(y_i - (\hat{y}_i^{t-1}+f_t(x_i)))^2 \\
    && g_i = \partial{\frac{(\hat{y}_i^{t-1} - y_i)^2}{\hat{y}^{t-1}}} = 2(\hat{y}^{t-1} - y_i) \\
    && h_i = \partial ^2{\frac{(\hat{y}_i^{t-1} - y_i)^2}{\hat{y}^{t-1}}} = 2
\end{eqnarray*}

由于在第t步$\hat{y}_i^{t-1}$是是一个已知的值,所以 $l(y_i,\hat{y}_i^{t-1})$ 是一个常数,其对函数优化不会产生影响,因此,可以将 $Obj^{(t)}$ 改写为
$$
Obj^{(t)} \approx \sum_{i=1}^n[g_if_t(x_i) + \frac{1}{2}h_if_t^2(x_i)] + \Omega (f_t) + constant
$$

所以我么只要求出每一步损失函数的一阶和二阶导的值(由于前一步的 $\hat{y}_i^{t-1}$ 是已知的,所以这两个值就是常数)代入上述等式,然后最优化目标函数,就可以得到每一步的$f(x)$ ,最后根据加法模型得到一个整体模型。

\section{未完待续:如何生成一颗新的树}

%\printbibliography
\bibliography{../ref}
\bibliographystyle{IEEEtran}
\end{document}
