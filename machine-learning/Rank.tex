\documentclass[12pt]{article}
%\usepackage[utf8]{inputenc}
%\documentclass[UTF8]{ctexart}
%\usepackage[UTF8, heading = false, scheme = plain]{ctex}
\usepackage{geometry}
%geometry{a4paper,scale=0.9}
\geometry{a4paper,left=1cm,right=1cm,top=1cm,bottom=2cm}
\usepackage{amsfonts}
\usepackage{color}
\usepackage{url}
%\usepackage{biblatex}
\usepackage{amsmath}
\usepackage{amssymb}
\usepackage{latexsym}
\usepackage{cite}
%\addbibresource{ref.bib}
%\bibliography{ref.bib}
\usepackage{caption}
\usepackage{graphicx, subfig}
\usepackage{float}
%\usepackage[fontset=ubuntu]{ctex}
%\usepackage{fontspec}
\usepackage{xeCJK}
%\usepackage[colorlinks,
%anchorcolor=black,
%citecolor=black]{hyperref}
%\setmainfont{SimSun}
\usepackage[section]{placeins}
\usepackage{enumitem}
\usepackage{framed}
\usepackage[framemethod=TikZ]{mdframed}
\usepackage{indentfirst}
\usepackage{setspace}%使用间距宏包
\linespread{1.5}
%\title{预备知识}
%\author{leolinuxer }
%\date{June 2020}
\title{LearningToRank 算法介绍}
%\author{leolinuxer }
%\date{June 2020}

\begin{document}
\maketitle

\section{RankNet\cite{From_RankNet_To_LambdaRank_To_LambdaMART}\cite{About_RankNet_LambdaRank}}
\subsection{符号定义}
输入的特征向量对:$x_i, x_j$;

对应的标注:$U_i, U_j$,$U_i \rhd U_j$ 代表 $U_i$ 应该排在 $U_j$ 前面。

Rank 的打分函数记作$s = f(x;w)$,模型的参数为 $w$。比较样本 $i,j$的打分函数记作:$S_{ij}$,并且 $S_{ij}$ 的取值空间为:$\{+1, -1, 0\}$,其中,$+1$ 代表 $i$ 的排序比 $j$ 靠前;

\subsection{代价函数和梯度}
RankNet和LambdaRank同属于pairwise方法。对于某一个query,pairwise方法并不关心某个doc与这个query的相关程度的具体数值,而是将对所有docs的排序问题转化为求解任意两个docs的先后问题,即:根据docs与query的相关程度,比较任意两个不同文档 $i$和$j$的相对位置关系,并将query更相关的doc排在前面。

RankNet巧妙的借用了 sigmoid 函数来定义 样本 $i$ 比样本 $j$ ($U_i \rhd U_j$)更相关的概率为:
$$
P_{ij} = P(U_i \rhd U_j) = \frac{1}{1 + e^{-\sigma(s_i - s_j)}}
$$

$\sigma$ 为待学习的参数,$\sigma(x) = wx+b$。

若$i$比$j$ 更相关,则$P_{ij} > 0.5$ ,反之 $P_{ij} < 0.5$。

所以,记 $\bar P_{ij}$为真实的概率(取值范围为 $[0,1]$),则有:
$$
\bar P_{ij} = \frac{1}{2}(1 + S_{ij})
$$

RankNet 使用交叉熵函数作为损失函数,单个样本对的交叉熵损失函数(loss)为:
$$
C_{ij} = -\sum_{i=1}^N\bar{y_{ij}}\log{y_{ij}} = -[\bar P_{ij}\log{P_{ij}} + (1-\bar P_{ij})\log{(1 - P_{ij})}]
$$

代入公示后,可以求得对于单个样本对的交叉熵损失函数具体表达式为
\begin{align}
C_{ij} &= -\bar P_{ij}\log{P_{ij}}  (1-\bar P_{ij})\log{1 - P_{ij}} \\
    &= -\frac{1}{2}(1+S_{ij})\cdot\log\frac{1}{1 + e^{-\sigma(s_i - s_j)}} - [1 - \frac{1}{2}(1+S_{ij})]\cdot\log{[1-\frac{1}{1 + e^{-\sigma(s_i - s_j)}}]} \\
    &= -\frac{1}{2}(1+S_{ij})\cdot\log\frac{1}{1 + e^{-\sigma(s_i - s_j)}} - \frac{1}{2}(1-S_{ij})\cdot[-\sigma(s_i-s_j)+\log{\frac{1}{1+e^{-\sigma(s_i-s_j)}}}] \\
    &= \frac{1}{2}(1-S_{ij})\cdot\sigma(s_i - s_j)+\log[1+e^{-\sigma(s_i-s_j)}
\end{align}

所以 $C_{ij}$ 关于任一待优化参数 $w_k$ 的偏导数为
$$
\frac{\partial C_{ij}}{\partial w_k} = \frac{\partial C_{ij}}{\partial s_i}\frac{\partial s_i}{\partial w_k} + \frac{\partial C_{ij}}{\partial s_j}\frac{\partial s_j}{\partial w_k}
$$

使用随机梯度下降法(SGD)对参数进行优化:
$$
w_k \rightarrow w_k - \eta\frac{C_{ij}}{\partial w_k} = w_k - \eta(\frac{\partial C_{ij}}{\partial s_i}\frac{\partial s_i}{\partial w_k} + \frac{\partial C_{ij}}{\partial s_j}\frac{\partial s_j}{\partial w_k})
$$

\subsection{应用}
根据上面的推导,给定两个样本 $i$ 和 $j$ ,可通过$s=f(x)$来比较它们排序的得分:
$$
P_{ij} = P(U_i \rhd U_j) = \frac{1}{1 + e^{-\sigma(s_i - s_j)}}
$$

但是实际应用时,分别计算各自如下得分即可:
$$
P_i = \frac{1}{1 + e^{-\sigma(s_i)}}
$$
$$
P_j = \frac{1}{1 + e^{-\sigma(s_j)}}
$$

原因如下:

如果 $U_i \rhd U_j$,那么有:
\begin{align}
P_{ij} &= P(U_i \rhd U_j) \rightarrow 1\\
    &\Rightarrow \frac{1}{1 + e^{-\sigma(s_i - s_j)}} \rightarrow 1 \\
    &\Rightarrow e^{-\sigma(s_i - s_j)} \rightarrow 0 \\
    &\Rightarrow w(x_i-w_j)+b \rightarrow \infty \\
    &\Rightarrow wx_i >> wx_j \\
    &\Rightarrow \frac{1}{1+e^{-\sigma s_i}} > \frac{1}{1+e^{-\sigma s_j}} \\
    &\Rightarrow P_i > P_j
\end{align}

因此,预测时不需要两两比较计算 $P_{ij}$,直接对各个样本$i$计算$P_i$,然后排序即可。

\bibliography{../ref}
\bibliographystyle{IEEEtran}
\end{document}

