\documentclass[12pt]{article}
%\usepackage[utf8]{inputenc}
%\documentclass[UTF8]{ctexart}
%\usepackage[UTF8, heading = false, scheme = plain]{ctex}
\usepackage{geometry}
%geometry{a4paper,scale=0.9}
\geometry{a4paper,left=1cm,right=1cm,top=1cm,bottom=2cm}
\usepackage{amsfonts}
\usepackage{color}
\usepackage{url}
%\usepackage{biblatex}
\usepackage{amsmath}
\usepackage{amssymb}
\usepackage{latexsym}
\usepackage{cite}
%\addbibresource{ref.bib}
%\bibliography{ref.bib}
\usepackage{caption}
\usepackage{graphicx, subfig}
\usepackage{float}
%\usepackage[fontset=ubuntu]{ctex}
%\usepackage{fontspec}
\usepackage{xeCJK}
%\usepackage[colorlinks,
%anchorcolor=black,
%citecolor=black]{hyperref}
%\setmainfont{SimSun}
\usepackage[section]{placeins}
\usepackage{enumitem}
\usepackage{framed}
\usepackage[framemethod=TikZ]{mdframed}
\usepackage{indentfirst}
\usepackage{setspace}%使用间距宏包
\linespread{1.5}
%\title{预备知识}
%\author{leolinuxer }
%\date{June 2020}

\title{各种场景的处理方法}
%\author{leolinuxer }
%\date{June 2020}

\begin{document}
\maketitle

\section{不均衡样本集的处理\cite{Handle_Unbalanced_Samples}}
\subsection{场景描述}
在训练二分类模型时,经常会遇到正负样本不均衡的问题,例如医疗诊断、网络入侵检测、信用卡反诈骗等。对于很多分类算法,如果直接采用不均衡的样本集来进行训练学习,会存在一些问题。例如,如果正负样本比例达到1:99,则分类器简单地将所有样本都判为负样本就能达到99\%的正确率,显然这并不是我们想要的,我们想让分类器在正样本和负样本上都有足够的准确率和召回率。

\subsection{问题}
对于二分类问题,当训练集中正负样本非常不均衡时,如何处理数据以更好地训练分类模型? 

\subsection{处理方法}
\subsubsection{基于数据的方法}
主要是对数据进行重采样,使原本不均衡的样本变得均衡。

直接的随机采样虽然可以使样本集变得均衡,但会带来一些问题:过采样对少数类样本进行了多次复制,扩大了数据规模,增加了模型训练的复杂度,同时也容易造成过拟合;欠采样会丢弃一些样本,可能会损失部分有用信息,造成模型只学到了整体模式的一部分。

为了解决上述问题,通常在过采样时并不是简单的复制样本,而是采用一些方法生成新的样本,这样可以降低过拟合的风险。

在实际应用中,具体的采样操作可能并不总是如上述几个算法一样,但基本思路很多时候还是一致的。例如,基于聚类的采样方法,利用数据的类簇信息来指导过采样/欠采样操作;经常用到的数据扩充 (data augmentation) 方法也是一种过采样,对少数类样本进行一些噪音扰动或变换(如图像数据集中对图片进行裁剪、翻转、旋转、加光照等)以构造出新的样本;而Hard Negative Mining则是一种欠采样,把比较难的样本抽出来用于迭代分类器。

\subsubsection{基于算法的方法}
在样本不均衡时,也可以通过改变模型训练时的目标函数(如代价敏感学习中不同类别有不同的权重)来矫正这种不平衡性;当样本数目极其不均衡时,也可以将问题转化为one-class learning / anomalydetection。本节主要关注采样,不再细述这些方法(我们会在其它章节的陆续推送相关知识点)。

%\printbibliography
\bibliography{../ref}
\bibliographystyle{IEEEtran}
\end{document}

